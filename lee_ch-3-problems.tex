\documentclass[11pt]{article}
\usepackage{amssymb}
\usepackage{amsthm}
\usepackage{enumitem}
\usepackage{amsmath}
\usepackage{bm}
\usepackage{adjustbox}
\usepackage{mathrsfs}
\usepackage{graphicx}
\usepackage{siunitx}
\usepackage[mathscr]{euscript}


\title{\textbf{Solutions to selected problems on Introduction to Topological Manifolds - John M. Lee.}}
\author{Franco Zacco}
\date{}

\addtolength{\topmargin}{-3cm}
\addtolength{\textheight}{3cm}

\newcommand{\N}{\mathbb{N}}
\newcommand{\Z}{\mathbb{Z}}
\newcommand{\Q}{\mathbb{Q}}
\newcommand{\R}{\mathbb{R}}
\newcommand{\HH}{\mathbb{H}}
\newcommand{\diam}{\text{diam}}
\newcommand{\cl}{\text{cl}}
\newcommand{\bdry}{\text{bdry}}
\newcommand{\inter}{\text{Int}}
\newcommand{\ext}{\text{Ext}}
\newcommand{\Pow}{\mathcal{P}}
\newcommand{\Topo}{\mathcal{T}}
\newcommand{\Or}{\text{ or }}
\newcommand{\setmin}{\setminus}


\theoremstyle{definition}
\newtheorem*{solution*}{Solution}

\begin{document}
\maketitle
\thispagestyle{empty}

\section*{Chapter 3 - New Spaces from Old}

\subsection*{Problems}

\begin{proof}{\textbf{3-1}}
    Let $M$ be an $n$-dimensional manifold with boundary, we want to show
    that $\partial M$ is an $(n-1)$-manifold (without boundary) when endowed
    with the subspace topology.

    We know that $M$ is a second countable and a Hausdorff space and since
    $\partial M$ is endowed with the subspace topology then it's
    a subspace of $M$ hence by Proposition 3.11 (d) and (f) we know that
    $\partial M$ is also second countable and Hausdorff subspace.

    On the other hand, let $p \in \partial M$ then since $M$ is a manifold
    with boundary there is a neighborhood $U \subseteq M$ of $p$ which is
    homeomorphic to an open set $V \subseteq \mathbb{H}^n$ i.e. there
    is a homeomorphism $\varphi: U \to V$. Since $p \in \partial M$ then
    $\varphi(p) \in V \cap \partial \mathbb{H}^n$ i.e.
    $\varphi(p) = (x_1, ..., x_n)$ with $x_n = 0$.
    So let us define
    $\varphi\big|_{U \cap \partial M}:
    U \cap \partial M \to V \cap \partial\mathbb{H}^n$
     as the restriction of $\varphi$
    to $U \cap \partial M$ we want to show it is a homeomorphism too.
    We know $\varphi\big|_{U \cap \partial M}$ bijective since
    $\varphi$ is bijective and since $\varphi$ is continuous then the
    restriction $\varphi\big|_{U \cap \partial M}$ is also continuous.
    Finally, since $\varphi^{-1}$ is continuous (given that $\varphi$ is
    a homeomorphism) then the restriction
    $\varphi^{-1}\big|_{U \cap \partial M}$ is also continuous hence
    $\varphi\big|_{U \cap \partial M}$ is a homeomorphism from
    $U \cap \partial M$ to $V \cap \partial \mathbb{H}^n$.
    
    This implies that every point of $\partial M$ has a neighborhood
    $U \cap \partial M$ homeomorphic to an open set
    $V \cap \partial \mathbb{H}^n$ in $\mathbb{R}^{n-1}$.
    
    Therefore $\partial M$ is an $(n-1)$-manifold without boundary.    
\end{proof}
\cleardoublepage
\begin{proof}{\textbf{3-3}}
    Let $X = \{0\} \cup \{1/n: n \in \N\}$ and $Y = \R$, and let
    $\{A_i\} = \{0\} \cup \{\{1/n\} : n \in \N\}$ be an infinite closed
    cover of $X$. Also, let us define $f_i:A_i \to \R$ as $f_0(0) = 1$ and
    and $f_i(1/i) = 0$.
    We want to prove that each $f_i$ is continuous.
    Let us take $U = (-1,1)$ for any $f_i$ such that $i \geq 1$ we see that
    $U$ is open in $\R$ and $f_i^{-1}(U)$ is open in $A_i$.
    Also, if we take $U =(0,2)$ for $f_0$ we see that $U$ is open in $\R$ and
    $f_0^{-1}(U)$ is open in $A_0$. Therefore each $f_i$ is continuous.

    Now, let us take $f:X \to \R$ such that $f|_{A_i} = f_i$, we want to prove
    $f$ is not continuous. Let $ U =(0,2) \subset \R$ which is open in $\R$ then
    $f^{-1}(U) = \{0\}$ but $\{0\}$ is not open in $X$.

    Therefore $f$ is not continuous and the Gluing Lemma does not need to hold
    when we consider an infinite closed cover.
\end{proof}
\begin{proof}{\textbf{3-6}}
    Let $X$ be a topological space and let
    $\Delta = \{(x,x): x \in X\} \subseteq X \times X$
    be the diagonal of $X \times X$.
    We want to prove that $X$ is Hausdorff if and only if $\Delta$ is closed.
    \begin{itemize}
        \item[$(\Rightarrow)$]
        Let $(x,y) \in X \times X \setmin \Delta$ and let $X$ be Hausdorff
        then there are open sets $U_x, U_y \subseteq X$ such that $x \in U_x$,
        $y \in U_y$ and $U_x \cap U_y = \emptyset$. Then we can build an open
        rectangle $U_x \times U_y$ such that no point of $\Delta$ is in
        $U_x \times U_y$ therefore for every point of $X \times X \setmin \Delta$
        there is a neighborhood which is contained in $X \times X \setmin \Delta$
        which implies that $\Delta$ is closed.

        \item[$(\Leftarrow)$] Let $\Delta$ be a closed set then
        $X \times X \setmin \Delta$ is open then every point of 
        $X \times X \setmin \Delta$ has a neighborhood contained in
        $X \times X \setmin \Delta$.
        Let $(x,y) \in X \times X \setmin \Delta$
        then there is an open rectangle
        $U_x \times U_y \subset X \times X \setmin \Delta$ where
        $U_x, U_y \subset X$ are open sets such that
        no point of $\Delta$ is in $U_x \times U_y$ so we have two neighborhoods
        $x \in U_x$ and $y \in U_y$ which are disjoint i.e.
        $U_x \cap U_y = \emptyset$.
        Therefore this implies that $X$ is Hausdorff.
    \end{itemize}
\end{proof}
\cleardoublepage
\begin{proof}{\textbf{3-10}}
    \begin{itemize}
        \item[]
        ($\Rightarrow$) Let $f: \coprod_{\alpha \in A} X_\alpha \to Y$
        be continuous, we want to prove that $f|_{X_\alpha}$ for
        each $\alpha \in A$ is continuous too.
    
        Let $U$ be an open set of $Y$ we know that $f^{-1}(U)$ is open in
        $\coprod_{\alpha \in A} X_\alpha$ then by the definition of disjoint
        union topology $f^{-1}(U) \cap X_\alpha$ is open in $X_\alpha$
        for $\alpha \in A$. Also, we know that
        \begin{align*}
            f|_{X_\alpha}^{-1}(U) 
            &= \{x \in X_\alpha : f|_{X_\alpha}(x) \in Y\}
            = \{x \in X_\alpha : f(x) \in Y\}
        \end{align*}
        and that
        \begin{align*}
            f^{-1}(U) 
            &= \{x \in \coprod_{\alpha \in A} X_\alpha : f(x) \in Y\}
        \end{align*}
        Therefore we see that $f|_{X_\alpha}^{-1}(U) = f^{-1}(U) \cap X_\alpha$
        and thus $f|_{X_\alpha}^{-1}(U)$ is open in $X_\alpha$
        which implies that $f|_{X_\alpha}$ is continuous.

        \item[]
        ($\Leftarrow$) Let $f|_{X_\alpha}: X_\alpha \to Y$
        be continuous for each $\alpha \in A$, we want to prove that
        $f: \coprod_{\alpha \in A} X_\alpha \to Y$ is continuous too.

        Let $U$ be an open set of $Y$ we know that $f|_{X_\alpha}^{-1}(U)$
        is open in $X_\alpha$ for every $\alpha \in A$.
        Also, we know that $f|_{X_\alpha}^{-1}(U) = f^{-1}(U) \cap X_\alpha$
        then by the definition of disjoint union topology, we see that
        $f^{-1}(U)$ must be open in $\coprod_{\alpha \in A} X_\alpha$
        and therefore $f$ is continuous.
    \end{itemize}
    Suppose now we define $\Topo$ to be the disjoint union topology and
    $\Topo'$ to be another topology with the Characteristic Property.

    Invoking the Characteristic Property of $\Topo'$ with
    $Y = \coprod_{\alpha \in A} X_\alpha$ in the disjoint union topology
    $\Topo$ shows that the identity map
    $$i: (\coprod_{\alpha \in A} X_\alpha)_{\Topo'}
    \to (\coprod_{\alpha \in A} X_\alpha)_\Topo$$
    is continuous and the same happens if we take 
    $Y = \coprod_{\alpha \in A} X_\alpha$ in $\Topo'$ and we apply the 
    Characteristic Property of $\Topo$ so the inverse is also continuous.
    Therefore the two toplogies are equal.
\end{proof}
\cleardoublepage
\begin{proof}{\textbf{3-12}}
\begin{itemize}
    \item [(a)] Let us define $\Topo_X$ to be the topology on $X$,
    let $\Topo_S = \{U \cap S : U\in \Topo_X\}$ be the subspace topology on $S$ and let $\Topo$ be
    the coarsest topology on $S$ for which $\iota_S: S \to X$ is continuous
    then by definition $\Topo$ is given by
    $\Topo = \{\iota_S^{-1}(U): U \in \Topo_X\}$.
    
    We want to show that $\Topo = \Topo_S$ and this will happen if
    $\iota_S^{-1}(U) = U \cap S$ for all $U \in \Topo_X$.

    Let $U \in \Topo_X$ then by definition
    $\iota_S^{-1}(U) = \{x \in S : \iota_S(x) \in U\}$
    hence we have that
    \begin{align*}
        \iota_S^{-1}(U) = \{x \in S : \iota_S(x) \in U\}
        = \{x \in S : x \in U\} = U \cap S
    \end{align*}
    Therefore $\Topo = \Topo_S$ and hence the subspace topology on $S$
    is the coarsest topology such that $\iota_S : S \to X$ is continuous.    

    \item [(b)] Let $A$ be finite so we will write $\prod_{i=1}^n X_i$
    instead of $\prod_{\alpha \in A} X_\alpha$.
    Let $\Topo$ be a topology on
    $\prod_{i=1}^n X_i$ for which the canonical projection
    $\pi_i: \prod_{i=1}^n X_i \to X_i$ is continuous.
    Also, let $\Topo_p$ be the product topology on
    $\prod_{i=1}^n X_i$, we want to show that $\Topo_p \subseteq \Topo$
    which implies that $\Topo_p$ is the coarsest topology where each $\pi_i$
    is continuous.

    By definition $\Topo$ is given by
    $\Topo = \bigcup_{i=1}^n \{\pi_i^{-1}(U_i) : U_i \text{ is open in} X_i\}$
    so we see that
    \begin{align*}
        \pi_i^{-1}(U_i) &= \{x \in \prod_{i=1}^n X_i : \pi_i(x) \in U_i\}\\
        &= \{x \in \prod_{i=1}^n X_i : x_i \in U_i\}\\
        &= X_1 \times ... \times U_i \times ... \times X_n
    \end{align*}
    Also, we see that there is a collection of open set $U_i \in X_i$ such that
    \begin{align*}
        U_1 \times ...\times U_i \times ... \times U_n
        \subseteq X_1 \times ... \times U_i \times ... \times X_n
    \end{align*}
    This implies that $\Topo_p \subseteq \Topo$ and therefore that $\Topo_p$
    is the coarsest topology on $\prod_{i=1}^n X_i$ where each
    $\pi_i: \prod_{i=1}^n X_i \to X_i$ is continuous. 
\cleardoublepage
    \item [(c)] Let $\Topo$ be a topology on $\coprod_{\alpha} X_\alpha$
    for which every canonical projection
    $\iota_\alpha: X_\alpha \to \coprod_{\alpha} X_\alpha$ is continuous.
    Also, let $\Topo_d$ be the disjoint union topology on
    $\coprod_{\alpha} X_\alpha$, we want to show that $\Topo \subseteq \Topo_d$
    which implies that $\Topo_d$ is the finest topology where each $\iota_\alpha$
    is continuous.

    Let $U$ be an open set of $\Topo$ then by definition each
    $\iota_\alpha^{-1}(U)$ is open in $X_\alpha$ but we see that
    \begin{align*}
        \iota_\alpha^{-1}(U) = \{x \in X_\alpha : \iota_\alpha(x) \in U \}
        = X_\alpha \cap U
    \end{align*}
    Then by definition of disjoint union topology, we have that 
    $U \subseteq \Topo_d$ which implies that $\Topo \subseteq \Topo_d$
    and therefore $\Topo_d$ is the finest topology on
    $\coprod_{\alpha} X_\alpha$ where each $\iota_\alpha$ is continuous.

    \item [(d)] Let $q: X \to Y$ be a surjective map and let $\Topo$ be a
    topology on $Y$ for which $q$ is continuous. Also, let $\Topo_q$ be the 
    quotient topology on $Y$, we want to show that $\Topo \subseteq \Topo_q$
    which implies that $\Topo_q$ is the finest topology for which $q$
    is continuous.

    Let $U$ be an open set of $\Topo$ then by definition since $q$ is
    continuous in this topololgy $q^{-1}(U)$ is open in $X$.
    In the oposite way if $q^{-1}(U)$ is open in $X$
    then $U$ must be open in $Y$ since $q$ is continuous.

    But by definition of quotient topology this implies that $U \in \Topo_q$
    which implies that $\Topo \subseteq \Topo_q$ and therefore $\Topo_q$
    is the finest topology on $Y$ where $q$ is continuous.
\end{itemize}
\end{proof}
\cleardoublepage
\begin{proof}{\textbf{3-13}}
    Let $f:X \to Y$ be a continuous map.
    \begin{itemize}
        \item [(b)] Let $f$ be a map that admits a continuous right inverse $g$.
        We want to show that $f$ is a quotient map.
        
        Let $U$ be an open set of $Y$ then $g(U)$ is open in $X$ since $f$
        is continuous.

        On the other hand, let $V = g(U) \subseteq X$ be an open set for
        some set $U$ in $Y$. Then we see that $f(V) = f(g(U)) = U$ since
        $f$ admits a right inverse $g$ which is continuous.
        
        Therefore $f$ is a quotient map.

        \item [(c)]
        Let us consider $f: [0,1) \to [0,1]$ such that $f(x) = x$, we see that
        $f$ is a topological embedding since it's injective, continuous and
        is a homeomorphism onto its image. Let $g: [0,1] \to [0,1)$ be the
        left inverse of $f$ we want to arrive at a contradiction.
        We know that $[0,1]$ is compact so if $g$ is
        continuous then $g([0,1])$ must be compact but $g([0,1]) = [0,1)$
        is not compact, a contradiction. Therefore $f$ has no continuous
        left inverse.
        
        Let us consider now $q:[0,1] \to \mathbb{S}^1$ such that
        $q(s) = e^{2\pi i s}$. In Example 3.66 we showed that $q$ is a
        quotient map. We want to show that there is no continuous map
        $r: \mathbb{S}^1 \to [0,1]$ such that $q(r(z)) = z$ for every
        $z \in \mathbb{S}^1$.
        From the definition of $q$ we see that both points $0,1 \in [0,1]$ are 
        sent to $(1,0) \in \mathbb{S}^1$ but $r((1,0))$ cannot be both
        $0,1 \in [0,1]$ so if we set $r((1,0)) = 0$ then we have a
        discontinuity at $1 \in [0,1]$ and the opposite happens if we set
        $r((1,0)) = 1$.
        Therefore $q$ doesn't have a continuous right inverse.
    \end{itemize}
\end{proof}
\cleardoublepage
\begin{proof}{\textbf{3-16}}
    Let $X$ be the subset $(\R\times \{0\})\cup (\R\times \{1\}) \subseteq \R^2$
    and let us define an equivalence relation on $X$ by declaring
    $(x,0)\sim(x,1)$ if $x \neq 0$. We want to show that the quotient space
    $X\setmin \sim$ is locally Euclidean and second countable, but not
    Hausdorff.
    \begin{itemize}
        \item
        Let us also define two maps $f: \R \to \R \times \{0\}$ such that
        $f(x) = (x,0)$ and $g:\R \to \R \times \{1\}$ such that $g(x) = (x,1)$.
        
        Then if we take a point $[(x,0)] \in X\setmin\sim$ and a neighborhood
        $U \subseteq X\setmin\sim$ of $[(x,0)]$
        then by definition $q^{-1}(U)$ is open and $f^{-1}(q^{-1}(U))$
        is open since $f$ is a homeomorphism hence $q\circ f$ is continuous.
        
        On the other hand if we let $x \in \R$ and $U \subseteq \R$ a neighborhood
        of $x$ then $f(U)$ is open since $f$ is a homeomorphism, also, $q(f(U))$
        is also open hence $(q \circ f)^{-1}$ is continuous.
        
        In the same way, if we take a point $[(x,1)] \in X\setmin\sim$ and a
        neighborhood $U \subseteq X\setmin\sim$ of $[(x,1)]$
        then by definition $q^{-1}(U)$ is open and $g^{-1}(q^{-1}(U))$
        is open since $g$ is a homeomorphism hence $q\circ g$ is continuous.
        
        And if we let $x \in \R$ and $U \subseteq \R$ a neighborhood
        of $x$ then $g(U)$ is open since $g$ is a homeomorphism, also, $q(g(U))$
        is also open hence $(q \circ g)^{-1}$ is continuous as well.
        
        Also, we see that both the map $q \circ f$ and $q \circ g$ are bijective
        from $\R$ to $X \setmin\sim$.
        
        Therefore for every neighborhood in $X\setmin\sim$
        there is a homeomorphism $q \circ f$ or $q \circ g$ to $\R$ and hence
        $X \setmin\sim$ is locally Euclidean.

        \item
        Given that $X = (\R \times \{0\}) \cup (\R \times \{1\})$ then $X$
        is second countable since it is composed of two copies of $\R$ and $\R$
        is second countable. In particular the countable basis $\mathcal{B}$
        for $X$ has to be the union of the countable basis of $\R \times \{0\}$
        and the countable basis of $\R \times \{1\}$ which we can assume
        are the same.
        
        Let us suppose we take a ball $B$ from the countable basis of
        $\R \times \{0\}$ then $q^{-1}(q(B))$ will contain the two balls
        from $\R \times \{0\}$ and $\R \times \{1\}$ and hence it's open
        which implies that $q(B)$ is open in $X\setmin\sim$ so 
        we can take the set of $\mathcal{B}' = \{q(B): B \in \mathcal{B}\}$
        as a countable basis for $X\setmin\sim$ and therefore 
        $X\setmin\sim$ is second countable.
\cleardoublepage
        \item
        Let us take $[(0,0)], [(0,1)] \in X\setmin\sim$ and let us suppose
        there is some neighborhoods $U_0$ and $U_1$ of $[(0,0)]$ and $[(0,1)]$
        respectively such that $U_0 \cap U_1 = \emptyset$, we want to arrive
        at a contradiction.

        By definition $q^{-1}(U_0)$ and $q^{-1}(U_1)$ are open in $X$
        but since part of $q^{-1}(U_0)$ is in $\R \times \{0\}$ and part of
        it is in $\R \times \{1\}$ because of the equivalence relation between
        them and the same thing happens for $q^{-1}(U_1)$ we have that
        $q^{-1}(U_0) \cap q^{-1}(U_1) \neq \emptyset$ then there must be a
        point that $(x,0) \in q^{-1}(U_0)$ and $(x,0) \in q^{-1}(U_1)$
        such that $q(x,0) \in U_0$ and $q(x,0) \in U_1$ and hence
        $U_0 \cap U_1 \neq \emptyset$, a contradiction.
        
        Therefore $X\setmin\sim$ is not Hausdorff.
    \end{itemize}
\end{proof}

\end{document}
















