\documentclass[11pt]{article}
\usepackage{amssymb}
\usepackage{amsthm}
\usepackage{enumitem}
\usepackage{amsmath}
\usepackage{bm}
\usepackage{adjustbox}
\usepackage{mathrsfs}
\usepackage{graphicx}
\usepackage{siunitx}
\usepackage[mathscr]{euscript}


\title{\textbf{Solutions to selected problems on Introduction to Topological Manifolds - John M. Lee.}}
\author{Franco Zacco}
\date{}

\addtolength{\topmargin}{-3cm}
\addtolength{\textheight}{3cm}

\newcommand{\N}{\mathbb{N}}
\newcommand{\Z}{\mathbb{Z}}
\newcommand{\Q}{\mathbb{Q}}
\newcommand{\R}{\mathbb{R}}
\newcommand{\diam}{\text{diam}}
\newcommand{\cl}{\text{cl}}
\newcommand{\bdry}{\text{bdry}}
\newcommand{\inter}{\text{Int}}
\newcommand{\ext}{\text{Ext}}
\newcommand{\Pow}{\mathcal{P}}
\newcommand{\Topo}{\mathcal{T}}
\newcommand{\Or}{\text{ or }}
\newcommand{\setmin}{\setminus}


\theoremstyle{definition}
\newtheorem*{solution*}{Solution}

\begin{document}
\maketitle
\thispagestyle{empty}

\section*{Chapter 2 - Topological Spaces}

\subsection*{Exercises}

\begin{proof}{\textbf{Exercise 2.2.}}
    \begin{itemize}
    \item [(a)] Let $X$ be any set such that every subset of $X$ is open.\\
    Let also $\Topo = \Pow(X)$ we want to prove $\Topo$ is a topology on $X$.
    \begin{itemize}
        \item [(i)] Given that $\Topo = \Pow(X)$ is the power set of $X$, by
        definition $X \in \Pow(X)$ and $\emptyset\in\Pow(X)$.
        \item [(ii)] Let $U_1, ..., U_n$ be a finite set of elements
        of $\Topo = \Pow(X)$ then by definition $U_1 \cap ...\cap U_n$ is
        a set of elements of $X$ and hence they are also in
        $\Pow(X) = \Topo$.
        \item [(iii)]Let $(U_{\alpha})_{\alpha \in A}$ be any (finite or
        infinite) family of elements of $\Topo = \Pow(X)$ then their union
        $\bigcup_{\alpha \in A} U_\alpha$ is the union of sets from $X$
        hence it's a subset of $X$ then they are also in $\Pow(X) = \Topo$.
    \end{itemize}
    \item [(b)] Let $Y$ be any set and $\Topo = \{Y,~\emptyset\}$
    we want to prove that $\Topo$ is a topology on $Y$.
    \begin{itemize}
        \item [(i)] By definition $Y$ and the $\emptyset$ are in $\Topo$.
        \item [(ii)] Any intersection between elements of
        $\Topo$ is either $Y$ or $\emptyset$ and both of them are in $\Topo$.
        \item [(iii)] Any union between elements of
        $\Topo$ is either $Y$ or $\emptyset$ and both of them are in $\Topo$.
    \end{itemize}
    \item [(c)] Let $Z = \{1,2,3\}$ and
    $\Topo = \{\{1\}, \{1,2\}, \{1,2,3\}, \emptyset\}$ we want to prove that
    $\Topo$ is a topology on $Z$.
    \begin{itemize}
        \item [(i)] By definition $Z$ and the $\emptyset$ are in $\Topo$.
        \item [(ii)]
        \begin{align*}
            \{1\} \cap \{1,2\} \cap \{1,2,3\} \cap \emptyset &= \emptyset\\
            \{1,2\} \cap \{1,2,3\} \cap \emptyset &= \emptyset\\
            \{1\} \cap \{1,2,3\} \cap \emptyset &= \emptyset\\
            \{1\} \cap \{1,2\} \cap \emptyset &= \emptyset\\
            \{1\} \cap \{1,2\} \cap \{1,2,3\} &= \{1\}\\
            \{1\} \cap \{1,2\} &= \{1\}\\
            \{1\} \cap \{1,2,3\} &= \{1\}\\  
            \{1\} \cap \emptyset &= \emptyset\\
            \{1,2\} \cap \{1,2,3\} &= \{1,2\}\\  
            \{1,2\} \cap \emptyset &= \emptyset\\  
            \{1,2,3\} \cap \emptyset &= \emptyset  
        \end{align*}
        Therefore every finite intersection of elements of $\Topo$ is in $\Topo$.
        \item [(iii)] 
        \begin{align*}
            \{1\} \cup \{1,2\} \cup \{1,2,3\} \cup \emptyset &= \{1,2,3\}\\
            \{1,2\} \cup \{1,2,3\} \cup \emptyset &= \{1,2,3\}\\
            \{1\} \cup \{1,2,3\} \cup \emptyset &= \{1,2,3\}\\
            \{1\} \cup \{1,2\} \cup \emptyset &= \{1,2\}\\
            \{1\} \cup \{1,2\} \cup \{1,2,3\} &= \{1,2,3\}\\
            \{1\} \cup \{1,2\} &= \{1,2\}\\
            \{1\} \cup \{1,2,3\} &= \{1,2,3\}\\  
            \{1\} \cup \emptyset &= \{1\}\\
            \{1,2\} \cup \{1,2,3\} &= \{1,2,3\}\\  
            \{1,2\} \cup \emptyset &= \{1,2\}\\  
            \{1,2,3\} \cup \emptyset &= \{1,2,3\}  
        \end{align*}
        Therefore every union of elements of $\Topo$ is in $\Topo$.
    \end{itemize}
    \end{itemize}
\end{proof}
\cleardoublepage
\begin{proof}{\textbf{Exercise 2.4.}}
\begin{itemize}
    \item [(a)]
    ($\Rightarrow$) Suppose $d$ and $d'$ generate the same topology on $M$ then
    the topologies $\Topo$ and $\Topo'$ generated by $d$ and $d'$ respectively
    have the same open sets.
    Let $x \in M$ and $r > 0$ then $B_r^{(d)}(x) \in \Topo$ and
    $B_r^{(d)}(x) \in \Topo'$ because $B_r^{(d)}(x)$ is an open set then there
    is $r_1 > 0$ such that $B_{r_1}^{(d')}(x) \subseteq B_r^{(d)}(x)$. In the
    same way $B_r^{(d')}(x) \in \Topo$ because it's an open set and hence
    then there is $r_2 > 0$ such that $B_{r_2}^{(d)}(x) \subseteq B_r^{(d')}(x)$.

    ($\Leftarrow$) Let $U \subset (M,d)$ be an open set. Let also
    $\Topo$ and $\Topo'$ be the topologies generated by $d$ and $d'$.
    It happens that $U \in \Topo$ since $U$ is open then there
    is some $r > 0$ such that $B_{r}^{(d)}(x) \subseteq U$ but also we know
    that there is some $r_1 > 0$ such that
    $B_{r_1}^{(d')}(x) \subseteq B_{r}^{(d)}(x) \subseteq U$ which implies
    that there is a ball around $x$ for $(M, d')$ and hence $U$ is also an open
    set in $(M,d')$ therefore $U \in \Topo'$. If we take now $U \subset (M,d')$
    such that $U \in \Topo'$ we can show in the same way that $U \in \Topo$
    which implies that $d$ and $d'$ generate the same topology on $M$.

    \item [(b)] Let $x \in M$ and $r>0$. By definition
    $B_{r}^{(d)}(x) = \{y \in M: d(x,y) < r\}$ then let $r_1 = rc > 0$ so
    $$B_{r_1}^{(d')}(x) = \{y \in M: cd(x,y) < rc\} = \{y \in M: d(x,y) < r\}$$
    Hence $B_{r}^{(d)}(x) = B_{r_1}^{(d')}(x)$.

    Now let $B_{r}^{(d')}(x) = \{y \in M: d'(x,y) < r\}$ if $r_2 = r/c > 0$
    we get that
    $$B_{r_2}^{(d)}(x) = \{y \in M: d(x,y) < r/c\} = \{y \in M: cd(x,y) < r\}$$
    Hence $B_{r}^{(d')}(x) = B_{r_2}^{(d)}(x)$.

    Therefore because of what we proved in (a) we see that $d$ and $d'$
    generate the same topology.

    \item [(c)] Let
    $$d(x,y) = |x - y| = \sqrt{(x_1 - y_1)^2 + ... + (x_n - y_n)^2}$$
    and
    $$d'(x,y) = \max\{|x_1 - y_1|, ..., |x_n - y_n|\}$$
    Let also $r > 0$ and $y \in B_r^d(x)$ then $d(x,y) < r$ but we know
    from problem B.1 that $d'(x,y) \leq d(x,y) \leq \sqrt{n}d'(x,y)$ then
    we see that $y \in B_{r}^{d'}(x)$ because
    $d'(x,y) \leq d(x,y) < r$ which imply that
    $B_r^d(x) \subseteq B_{r}^{d'}(x)$.

    On the other hand, let $y \in B_{r_1}^{d'}(x)$ where $r_1 = r/\sqrt{n}$
    then$\sqrt{n}d'(x,y) < r$ and from problem B.1 we have that
    $d(x,y) \leq \sqrt{n} d'(x,y) < r$
    which implies that $y \in B_{r}^{d}(x)$ and hence
    $B_{r_1}^{d'}(x) \subseteq B_{r}^{d}(x)$.
    With this and the result we got from problem (a) we get that $d$ and
    $d'$ generate the same topology.

    \item [(d)] Let $x \in X$ and $0 < r \leq 1$ then $B_r^d(x) = \{x\}$
    so $\{x\}$ is an open set. Now let $S \subset X$ be any set of $X$ we see
    that it can be written as
    $$S = \bigcup_{x \in X} B_r^d(x)$$
    Which is also open. Hence the topology induced by $d$ is a topology where
    every set of $X$ is open then the topology is the discrete topology.

    \item [(e)] We know that the discrete metric on $\Z$ generates the discrete
    topology on $\Z$ so we want to show that the Euclidean metric generates
    it as well. It is sufficient to prove that the singletons are open sets in
    $\Z$ with the Euclidean metric from there we can generate any set as we did
    for the discrete metric hence they generate the same topology.
    Let $x \in \Z$ then $\{x\}$ is open since there is $0\leq r\leq 1$ such
    that $B_r^d(x) \subseteq \{x\}$ where $d$ is the Euclidean metric.
 
\end{itemize}
\end{proof}
\begin{proof}{\textbf{Exercise 2.5.}}
    We want to show that
    $$\Topo = \{Z : Z \subset Y \text{ and $Z$ is open on $X$}\}$$
    is a topology on $Y$.
    \begin{itemize}
    \item [(i)] Given that $X$ is a topological space then $\emptyset \subset X$.
    And since $\emptyset$ is open it is also a subset of $Y$ so
    $\emptyset \subset \Topo$. Also, by definition $Y$ is open in $X$ 
    then $Y \subset \Topo$.
    \item [(ii)] Given that $X$ is a topological space then any intersection
    of finitely many open subsets of $X$ is an open subset of $X$. Then since
    $Y \subset X$ then any finite intersection of open subsets of $Y$ is an open
    subset of $Y$.
    \item [(iii)] Given that $X$ is a topological space then any union of
    arbitrarily many open subsets of $X$ is an open subset of $X$. In
    particular, since $Y \subset X$ then any union of
    arbitrarily many open subsets of $Y$ is an open subset of $Y$.
    \end{itemize}
    Therefore $\Topo$ is a topology on $Y$.
\end{proof}
\cleardoublepage
\begin{proof}{\textbf{Exercise 2.6.}}
    Let $\{\Topo_\alpha\}_{\alpha \in A}$ be a collection of topologies on $X$.
    We want to prove that $\Topo = \bigcap_{\alpha \in A} \Topo_\alpha$ is also
    a topology on $X$.
    \begin{itemize}
    \item [(i)] Given that every $\Topo_\alpha$ is a topology on $X$ then
    $\emptyset \in \Topo_\alpha$ and $X \in \Topo_\alpha$ for every
    $\alpha \in A$ therefore $\emptyset \in \Topo$ and $X \in \Topo$.
    \item [(ii)] Let $U_1, ..., U_n$ be a finite set of elements from $\Topo$
    then each $U_i$ is an open set because it belongs to $\Topo_\alpha$
    for every $\alpha \in A$
    also $U_1 \cap ... \cap U_n$ is open and is in every $\Topo_\alpha$ because
    they are topologies on $X$. Therefore $U_1 \cap ... \cap U_n \in \Topo$.    
    \item [(iii)] Let $(U_{\beta})_{\beta \in A}$ be any family of elements
    of $\Topo$ as before each $U_\beta$ is in every $\Topo_\alpha$ by the
    definition of $\Topo$ and since $\Topo_\alpha$ are topologies on $X$
    it must happen that $\bigcup_{\beta \in A} U_\beta \in \Topo_\alpha$
    hence $\bigcup_{\beta \in A} U_\beta \in \Topo$. 
    \end{itemize}
    Therefore $\Topo$ is a topology on $X$.
\end{proof}
\begin{proof}{\textbf{Exercise 2.9.}}
    We will prove Proposition 2.8.
    \begin{itemize}
        \item [(a)]
        ($\Rightarrow$) Let $x \in \inter(A)$ since $\inter(A)$ is open
        and $\inter(A) \subseteq A$
        then $x$ has a neighborhood contained in $A$.

        ($\Leftarrow$) Let $U \subseteq A$ be an open neighborhood that
        contains a point $x \in A$ then since $\inter(A)$ is the largest
        open subset contained in $A$ it must happen that
        $U \subseteq \inter(A)$. Therefore
        $x \in \inter(A)$.

        \item [(b)]
        ($\Rightarrow$) Let $x \in \ext(A)$ then by definition
        $x \in X \setminus \overline{A}$ which is an open set
        and we see that $X \setminus \overline{A} \subseteq X \setminus A$
        hence $x$ has a neightborhood contained in $X\setminus A$.

        ($\Leftarrow$) Let $U \subseteq X \setmin A$ be an open neighborhood
        that contains a point $x \in X \setmin A$
        then since $\inter(X \setmin A)$ is
        the largest open subset contained in $X \setmin A$ it must happen that
        $U \subseteq \inter(X \setmin A)$ and also 
        we know that $\inter(X \setmin A) = X \setmin \overline{A}$ hence 
        $x \in X \setmin \overline{A} = \ext(A)$.

        \item [(c)]
        ($\Rightarrow$) Let $x \in \partial A$ and $U$ be a neighborhood of $x$
        since $x \not\in \inter(A)$ then $U \not\subseteq A$ so there is a
        point $y \in U$ such that $y \in X \setmin A$. On the other hand,
        since $x \not\in \ext(A)$ then $U \not\subseteq X \setmin A$ so there
        is a point $z \in U$ such that $z \in A$.

        ($\Leftarrow$) Let $U$ be a neighborhood of some point $x$ we know
        there is a point $y \in U$ such that $y \in X \setmin A$ then
        $U \not\subseteq A$ hence $x \not\in \inter(A)$. Also, we know that 
        there is a point $z \in U$ such that $z \in A$ then
        $U \not\subseteq X \setmin A$ hence $x \not\in \ext(A)$. Therefore
        this implies that $x \in \partial A$.

        \item [(d)]
        ($\Rightarrow$) Let $x \in \overline{A}$ and $U$ be a neighborhood of
        $x$ since $x \not\in \ext(A) = X \setmin \overline{A}$ then
        $U \not\subseteq X \setmin A$ so there is a point $y \in U$
        such that $y \in A$.

        ($\Leftarrow$) Let $x \in U$ be a neighborhood of $x \in A$ then
        $U \not\subseteq X \setmin A$ hence
        $x \not\in \ext(A) = X \setmin \overline{A}$. Therefore
        $x \in \overline{A}$.

        \item [(e)]
        Let $x \in \overline{A}$ given that $A \subseteq \overline{A}$ then
        $x$ might be in $A$ as well. Suppoose $x \not\in A$ hence
        $x \not\in \inter(A)$ but also we know by definition that
        $x \not\in \ext(A) = X \setmin \overline{A}$ therefore $x$ must be in
        $\partial A$. Hence $\overline{A} = A \cup \partial A$.

        In the same way, let $x \in \overline{A}$ given that
        $\inter(A) \subset \overline{A}$ then $x$ might be in $\inter(A)$ as
        well but let us suppose that $x \not\in \inter(A)$ but also we know by
        definition that $x \not\in \ext(A) = X \setmin \overline{A}$ therefore
        $x$ must be in $\partial A$. Hence we also have that
        $\overline{A} = \inter(A) \cup \partial A$.

        \item [(f)] By definition $\inter(A)$ is the largest open subset
        contained in $A$ hence it's open in $X$. Also, we know that
        $\overline{A}$ is closed then $X \setmin \overline{A} = \ext(A)$
        is open in $X$.

        By definition $\overline{A}$ is the smallest closed subset containing
        $A$ hence it's closed in $X$. Also, we know that $\inter(A) \cup \ext(A)$
        is open because they are both open as we proved before then
        $\partial A = X \setmin (\inter(A) \cup \ext(A))$ is closed in $X$.

        \item [(g)]
        Suppose $A$ is open in $X$ then $A$ is the largest open subset contained
        in $A$ hence $A = \inter(A)$.

        If $A = \inter(A)$ then no element of $A$ is in $\ext(A)$ or in
        $\partial A$ hence $A$ contains none of its boundary points.
        
        If $A$ contains none of its boundary points then no neighborhood
        contains a point of $X \setmin A$ hence every point of $A$ has a
        neighborhood contained in $A$.

        Finally, if every point of $A$ has a neighborhood contained in $A$
        then it must happen that $A = \inter(A)$ which we know is open.
        Therefore $A$ is open in $X$.

        \item [(h)]
        Suppose $A$ is closed in $X$ then $A$ is the smallest closed subset
        that contains $A$ hence $A = \overline{A}$.

        If $A = \overline{A}$ then $A = \inter{A} \cup \partial A$ hence $A$
        contains all of its boundary points.

        Let $A$ contain all of its boundary points. Let us note that
        $\overline{A} = \inter(A) \cup \partial A$ since
        $\partial A \subseteq A$ and $\inter(A) \subseteq A$ by definition
        then we have that $\overline{A} \subseteq A$ and hence
        $A$ is closed which implies that $X \setmin A$ is open and
        therefore every point of $X \setmin A$ has a neighborhood contained in
        $X \setmin A$. 

        Finally, if every point of $X \setmin A$ has a neighborhood contained
        in $X \setmin A$ then $X \setmin A$ is open and hence $A$ is closed.
    \end{itemize}
\end{proof}
\cleardoublepage
\begin{proof}{\textbf{Exercise 2.10.}}
    
    ($\Rightarrow$) Let $A \subseteq X$ be a closed subset in a topological space
    $X$ also let $p \in X$ be a limit point of $A$ we want to prove that also
    $p \in A$. Since every neighborhood $U$ of $p$ contains a point of $A$ then
    $U \cap A \neq \emptyset$ hence $U \not\subseteq X \setmin A$
    and since $A$ is closed then every point of $X \setmin A$ must contain a
    neighborhood contained in $X \setmin A$ hence $p \not\in X \setmin A$
    which implies that $p \in A$. Therefore $A$ contains all of its
    limit points.
    
    ($\Leftarrow$) Let $A \subseteq X$ be a subset in a topological space
    $X$ that contains all of its limit points, we want to prove that $A$ is
    closed. Suppose that $X \setmin A$ is not open we want to arrive at a
    contradiction. Let $p \in X \setmin A$ and let $U$ be a neighborhood of
    $p$ since $X \setmin A$ is not open then for every $U$ of $p$ we have that
    $U \cap A \neq \emptyset$ then $p$ is a limit point of $A$
    but $A$ contains all of its limits point which is a contradiction.
    Therefore $X \setmin A$ is open which implies that $A$ is closed.
    
\end{proof}
\begin{proof}{\textbf{Exercise 2.11.}}

    ($\Rightarrow$) Let $x \in \overline{A} = X$ then every neighborhood $U$
    where $x \in U$ contains a point of $A$ because of Proposition 2.8.(d)
    therefore every non-empty open subset of $X$ contains a point of $A$.

    ($\Leftarrow$) Let us suppose $\overline{A} \neq X$ we want to arrive at
    a contradiction. Let $x \in X$ such that $x \not\in \overline{A}$ but we
    know that for every neighborhood $U$ where $x \in U$ there is a point of
    $A$ in it but then $x \in \overline{A}$ because of Proposition 2.8.(d) so
    we have a contradiction and therefore must be that $\overline{A}=X$.
\end{proof}
\begin{proof}{\textbf{Exercise 2.12.}}
    Let $X$ be a topological metric space then from the topological convergence
    definition if $(x_i)_{i=1}^\infty$ is a sequence that converges to $x \in X$
    we know that for every neighborhood $U$ of $x$ there exists $N \in \N$ such
    that $x_i \in U$ for all $i \geq N$ but since we are in a metric space
    we have that for every neighborhood there is a ball
    $B_\epsilon(x) \subseteq U$ for some $\epsilon > 0$ which implies that
    there is $N' \in \N$ with $N'\geq N$ such that when $i \geq N'$ we have
    that $d(x_i, x) < \epsilon$. Therefore when $X$ is a topological metric
    space the two definitions are equivalent.
\end{proof}
\begin{proof}{\textbf{Exercise 2.13.}}
    Let $(x_i)$ be a convergent sequence in the discrete topological space $X$.
    Hence there is some $x \in X$ such that for every neighborhood $U$ of $x$
    there exists $N \in \N$ such that $x_i \in U$ for all $i \geq N$.
    Since $X$ is a discrete topological space this implies that the set $\{x\}$
    is an open set and also a neighborhood around $x$ so there exists some
    $N \in \N$ such that $x_i \in \{x\}$ for all $i \geq N$ but this implies that
    $x_i = x$ for every $i \geq N$. Therefore convergent sequences in discrete
    topological spaces are eventually constant.
\end{proof}
\cleardoublepage
\begin{proof}{\textbf{Exercise 2.14.}}
    Let $A \subseteq X$ and $(x_i) \subseteq A$ such that $x_i \to x$ where
    $x \in X$ then by the definition of a convergent sequence we have that
    for every neighborhood $U$ of $x$ there is some $N \in \N$ such that when
    $i \geq N$ we have that $x_i \in U$ this implies that every neighborhood
    of $x$ has at least a point of $A$ therefore $x \in \overline{A}$.
\end{proof}
\begin{proof}{\textbf{Exercise 2.16.}}
    
    ($\Rightarrow$)
    Let $f:X \to Y$ be a continuous map and let $V \subseteq Y$ be closed
    subset then $Y \setmin V$ is open and since $f$ is continuous we have that
    $f^{-1}(Y \setmin V) = X \setmin f^{-1}(V)$ is also open then
    $$X \setmin (X\setmin f^{-1}(V))
    = (f^{-1}(V) \cap X) \cup (X \setmin X)
    = f^{-1}(V)$$
    is closed.

    ($\Leftarrow$)
    Let $U \subseteq Y$ be an open subset then $Y \setmin U$ is closed so we
    have that $f^{-1}(Y \setmin U) = X \setmin f^{-1}(U)$ is also closed hence
    $$X \setmin (X\setmin f^{-1}(U))
    = (f^{-1}(U) \cap X) \cup (X \setmin X)
    = f^{-1}(U)$$
    is open which implies that $f$ is continuous.
\end{proof}
\begin{proof}{\textbf{Exercise 2.18.}}
    \begin{itemize}
    \item [(a)] Let $f: X \to Y$ be a constant map where every $x \in X$ is
    sent to a constant $y \in Y$ and let $U \subseteq Y$ be an open set then
    if $y \not \in U$ we have that $f^{-1}(U) = \emptyset$ which is an open set
    if $y \in U$ then $f^{-1}(U) = X$ which is also an open set, therefore $f$
    is continuous.

    \item [(b)] Let ${Id}_X: X \to X$ be the identity map and let
    $U \subset X$ be an open set hence $Id_X^{-1}(U) = U$ since $Id_X$ is the
    identity map, hence $Id_X^{-1}(U)$ is also an open set which implies that
    $Id_X$ is continuous.

    \item [(c)] Let $f:X \to Y$ be a continuous function and let
    $f_U: U \to Y$ be a restriction of $f$ to an open subset $U \subset X$ also
    let $V \subset Y$ be an open set from $Y$ then we see that
    $f_U^{-1}(V) = f^{-1}(V) \cap U$ and we know that $f^{-1}(V)$ is open since
    $f$ is continuous and $U$ is open by definition then $f^{-1}_U(V)$ is also
    an open set.
    Therefore $f_U$ the restriction of $f$ to an open set $U \subset X$ is
    continuous.
    \end{itemize}
\end{proof}
\cleardoublepage
\begin{proof}{\textbf{Exercise 2.20.}}
    We want to prove that "homeomorphic" is an equivalence relation on the
    class of all topological spaces then
    \begin{itemize}
        \item [(a)] Let $X$ be a topological space then there is the identity
        map $Id_X: X \to X$ which is continuous as we saw in Proposition 2.17(b)
        and bijective by definition. Also, we have that $Id_X = Id_X^{-1}$
        hence $Id_X^{-1}$ is also continuous. Therefore $X$ is homeomorphic
        to $X$ i.e. $X \approx X$.

        \item [(b)] Let $X \approx Y$ i.e. $X$ is homeomorphic to $Y$ then
        there is $f:X \to Y$ such that $f$ is bijective and continuous and also
        $f^{-1}$ is continuos. Now let us define $g = f^{-1}$ hence
        $g: Y \to X$ where $g$ is bijective and continuous by definition and
        $g^{-1} = (f^{-1})^{-1} = f$ is also continuos. Therefore $Y \approx X$.

        \item [(c)] Let $f:X\to Y$ and $g:Y \to Z$ be homeomorphisms
        (i.e. $X \approx Y$ and $Y \approx Z$) then there is $h = g \circ f$
        such that $h: X \to Z$ which we know is continuous because of
        Proposition 2.17(d) and is bijective since $f$ and $g$ are bijective.
        Also, if we define $h^{-1} = f^{-1}\circ g^{-1}$ we have that
        $h^{-1}:Z \to X$ and $h^{-1}$ is continuos since $f^{-1}$ and $g^{-1}$
        are continuous and their composition is continuous. Therefore
        if $X \approx Y$ and $Y \approx Z$ then $X \approx Z$.
    \end{itemize}
    Finally since "homeomorphic" is a reflexive, symmetric and transitive
    relation then "homeomorphic" is an equivalence relation on the class of 
    all topological spaces.
\end{proof}
\begin{proof}{\textbf{Exercise 2.21.}}
    Let $(X_1, \Topo_1)$ and $(X_2, \Topo_2)$ be topological spaces and $f:X_1 \to X_2$
    a bijective map.

    ($\Rightarrow$) Let $f$ be a homeomorphism, we want to prove that
    $f(\Topo_1) = \Topo_2$. Let $U \in \Topo_1$ i.e. $U$ is an open set from
    $X_1$ since $f$ is a homeomorphism then $f^{-1}$ is continuous hence
    $(f^{-1})^{-1}(U)$ is an open set of $X_2$ so $(f^{-1})^{-1}(U) \in \Topo_2$ 
    but $(f^{-1})^{-1}(U) = f(U) \in \Topo_2$ because $f$ is bijective.

    On the other hand, let us name $V = f(U) \in \Topo_2$ since $f$ is a
    homeomorphism then $f$ is continuous so $f^{-1}(V) = f^{-1}(f(U)) = U$ is
    an open set from $X_1$ hence $U \in \Topo_1$.
    
    Therefore $U \in \Topo_1$ if and only if $f(U) \in \Topo_2$ i.e.
    $f(\Topo_1) = \Topo_2$.

    ($\Leftarrow$) Let $U \in \Topo_1$ we know that $f(U) \in \Topo_2$. Let us
    name $V = f(U)$ then $f^{-1}(V) = U$ because $f$ is bijective and
    $U \in \Topo_1$ hence $f^{-1}(V)$ is an open set of $X_1$ which implies
    that $f$ is continuous.
    
    Let $U \in \Topo_1$ then $(f^{-1})^{-1}(U) = f(U)$ because $f$ is bijective
    but we know that $f(U) \in \Topo_2$ hence $(f^{-1})^{-1}(U)$ is an open set
    of $X_2$ which implies that $f^{-1}$ is continuous.

    Finally, since we know that $f$ is bijective we get that $f$ is a
    homeomorphism.
\end{proof}
\cleardoublepage
\begin{proof}{\textbf{Exercise 2.22.}}
    Let $f:X \to Y$ be a homeomorphism and let $U \subseteq X$ be an open
    subset. Since $f$ is a homeomorphism then $f^{-1}$ is continuous hence
    $(f^{-1})^{-1}(U)$ is an open set of $Y$ but $(f^{-1})^{-1}(U) = f(U)$
    because $f$ is bijective therefore $f(U)$ is an open set of $Y$.

    Let $f|_U: U \to f(U)$ be the restriction of $f$ to $U$. Since $f$ is a
    homeomorphism then $f$ is continuous and we know from Exercise 2.18 (c)
    that then $f|_U$ is continuous. Let also $f|_U^{-1}:f(U) \to U$ we know
    $f^{-1}$ is continuous since $f$ is a homeomorphism then because of
    Exercise 2.18 (c) we have that $f|_U^{-1}$ is also continuous.
    Finally, since $f$ is biyective then $f|_U$ is also biyective.
    Therefore $f|_U$ is also a homeomorphism.
\end{proof}
\begin{proof}{\textbf{Exercise 2.23.}}

    ($\Rightarrow$) Let $Id_X: (X,\Topo_1) \to (X,\Topo_2)$ be a continuous
    identity map then let $U \subseteq (X,\Topo_2)$ since $Id_X$ is continuous
    we have that $Id_X^{-1}(U) \subseteq (X,\Topo_1)$ but $Id_X^{-1}(U) = U$
    which implies that $\Topo_2 \subseteq \Topo_1$.

    ($\Leftarrow$) Let $\Topo_1$ be finer than $\Topo_2$ i.e.
    $\Topo_2 \subseteq \Topo_1$ now let $U \subseteq (X, \Topo_2)$
    then $U \subseteq (X, \Topo_1)$ but we know that $U = Id_X^{-1}(U)$ hence
    $Id_X^{-1}(U) \subseteq (X, \Topo_1)$. Therefore $Id_X$ is continuous.

    ($\Rightarrow$) Let $Id_X: (X,\Topo_1) \to (X,\Topo_2)$ be a homeomorphism
    then $Id_X$ is continuous which implies from what we just proved that
    $\Topo_2 \subseteq \Topo_1$, on the other hand we know that $Id_X^{-1}$ is
    also continuous hence we also get that $\Topo_1 \subseteq \Topo_2$ which
    implies that $\Topo_1 = \Topo_2$.

    ($\Leftarrow$) Let $\Topo_1 = \Topo_2$ hence we can write that
    $\Topo_2 \subseteq \Topo_1$ which implies that $Id_X$ is continuous
    from what we just proved, also, if we write $\Topo_1 \subseteq \Topo_2$
    this implies that $Id_X^{-1}$ is continuous. Finally, we know that $Id_X$
    is bijective by definition, therefore $Id_X$ is a homeomorphism.
\end{proof}
\cleardoublepage
\begin{proof}{\textbf{Exercise 2.27.}}
    Let us first check that
    \begin{align*}
        \varphi^{-1}(x,y,z) = \frac{(x,y,z)}{\max\{|x|,|y|,|z|\}}
    \end{align*}
    is the inverse of $\varphi$ as defined in example 2.26. so we compute the
    following
    \begin{align*}
        \varphi(\varphi^{-1}(x,y,z)) &=
        \frac{
            (\frac{x}{\max\{|x|,|y|,|z|\}},
            \frac{y}{\max\{|x|,|y|,|z|\}},
            \frac{z}{\max\{|x|,|y|,|z|\}})
        }{
            \sqrt{
                \frac{x^2}{\max\{|x|,|y|,|z|\}^2} +
                \frac{y^2}{\max\{|x|,|y|,|z|\}^2} +
                \frac{z^2}{\max\{|x|,|y|,|z|\}^2}
            }
        }\\
        &= \frac{\frac{(x, y, z)}{\max\{|x|,|y|,|z|\}}}{
            \frac{\sqrt{x^2 + y^2 + z^2}}{\max\{|x|,|y|,|z|\}}}\\
        &=\frac{(x, y, z)}{\sqrt{x^2 + y^2 + z^2}} 
    \end{align*}
    But since $(x,y,z)$ represents the unit sphere $\mathbb{S}^2$ then
    $\sqrt{x^2 + y^2 + z^2} = 1$ hence $\varphi(\varphi^{-1}(x,y,z)) = (x,y,z)$.
    Now we prove the same for the opposite composition
    \begin{align*}
        \varphi^{-1}(\varphi(x,y,z)) &=
        \frac{
            (\frac{x}{\sqrt{x^2 + y^2 + z^2}},
            \frac{y}{\sqrt{x^2 + y^2 + z^2}},
            \frac{z}{\sqrt{x^2 + y^2 + z^2}})
        }{
            \max\bigg\{
                \bigg|\frac{x}{\sqrt{x^2 + y^2 + z^2}}\bigg|,
                \bigg|\frac{y}{\sqrt{x^2 + y^2 + z^2}}\bigg|,
                \bigg|\frac{z}{\sqrt{x^2 + y^2 + z^2}}\bigg|,
            \bigg\}
        }\\
        &= \frac{\frac{(x, y, z)}{\sqrt{x^2 + y^2 + z^2}}}{
            \frac{\max\{|x|,|y|,|z|\}}{\sqrt{x^2 + y^2 + z^2}}}\\
        &=\frac{(x,y,z)}{\max\{|x|,|y|,|z|\}}
    \end{align*}
    In the same way here $(x,y,z)$ represents the unit cube $\bf{C}$ hence\\
    $\max\{|x|, |y|, |z|\} = 1$ therefore
    $\varphi^{-1}(\varphi(x,y,z)) = (x,y,z)$ as we wanted.

    Finally, we want to prove that $\varphi^{-1}: \mathbb{S}^2 \to \bf{C}$ is
    a continuous function.
    Since $\varphi^{-1}(x,y,z) = (x,y,z)/\max\{|x|, |y|, |z|\}$ then
    $\varphi^{-1}$ is the division between two functions the identity function
    $Id: (x,y,z) \to (x,y,z)$ and the infinity norm
    $\|(x,y,z)\|_\infty:(x,y,z) \to \max\{|x|, |y|, |z|\}$ where both of them
    are continuous functions therefore $\varphi^{-1}$ is continuous.
\end{proof}
\cleardoublepage
\begin{proof}{\textbf{Exercise 2.28.}}
    Let $a(s) = e^{2\pi i s} = \cos(2\pi s) + i\sin(2\pi s)$ we want to show
    first that $a$ is continuous then let $\epsilon >0$ we want to show that
    $\|a(s) - a(t)\|_2 < \epsilon$  whenever $\|s - t\|_2 < \delta$. Let
    us suppose $\|a(s) - a(t)\|_2 < \epsilon$ is true then we have that
    \begin{align*}
        \sqrt{(\cos(2\pi s) - \cos(2\pi t))^2 + (\sin(2\pi s) - \sin(2\pi t))^2} &< \epsilon\\
        \sqrt{2 - 2\cos(2\pi s)\cos(2\pi t) - 2\sin(2\pi s)\sin(2\pi t)} &< \epsilon\\
        \sqrt{2 - 2\cos(2\pi (s -t))} &< \epsilon\\
        1 + \cos(\pi - 2\pi (s-t)) &< \frac{\epsilon^2}{2}\\
        -2\pi (s-t) &< \arccos\bigg(\frac{\epsilon^2}{2} - 1\bigg) - \pi\\        
        -(s-t) &< \frac{1}{2\pi}\arccos\bigg(\frac{\epsilon^2}{2} - 1\bigg) - \frac{1}{2}
        % \sqrt{(t-s)^2} < \frac{1}{2\pi}\arccos\bigg(\frac{\epsilon^2}{2} - 1\bigg) - \frac{1}{2}\\
    \end{align*}
    Therefore if we take $\delta = \frac{1}{2\pi}\arccos\bigg(\frac{\epsilon^2}{2} - 1\bigg) - \frac{1}{2}$
    whenever $$\|s-t\|_2 = \sqrt{(s-t)^2} < \delta$$
    we get that $\|a(s) - a(t)\|_2 < \epsilon$ which implies that $a$ is
    continuous.

    Let us prove now that $a$ is one-to-one so let us suppose $a(s) = a(t)$
    then $e^{2\pi i s} = e^{2\pi i t}$ which implies that $s = t$ so $a$ is
    one-to-one.

    Now we want to prove $a$ is onto, let $v \in \mathbb{S}^1$ we want to show
    that there is some $s \in [0,1)$ such that $v = a(s)$ let us take
    $s = \frac{\log(v)}{2\pi i}$ then
    \begin{align*}
        a(s) &= e^{2\pi i (\frac{\log(v)}{2\pi i})} = e^{\log(v)} = v
    \end{align*}
    Therefore for every $v \in \mathbb{S}^1$ there is some $s \in [0,1)$ such
    that $a(s) = v$ which implies that $a$ is onto.

    Finally, we want to show that $a^{-1}$ is not continuous. Let
    $x_n = \cos(2\pi (1 + 1/n)) + i\sin(2\pi(1 + 1/n))$ be a sequence in
    $\mathbb{S}^1$ which tends to $1 \in \mathbb{S}^1$ and we see that
    $(1 + 1/n) \to 1$ but $1 \not\in [0,1)$ therefore $a^{-1}$ is not
    continuous.
\end{proof}
\cleardoublepage
\begin{proof}{\textbf{Exercise 2.29.}}
\begin{itemize}
    \item [$(a) \Rightarrow (b)$] Let $f$ be a homeomorphism then if
    $U \subseteq X$ is an open set we know that $f(U) = (f^{-1})^{-1}(U)$ is
    open since $f^{-1}$ is continuous.
    \item [$(b) \Rightarrow (c)$] Let $f$ be open. If $E \subseteq X$ is a
    closed set then $X \setminus E$ is open and since $f$ is open
    then $f(X \setminus E) = Y \setminus f(E)$ is open hence
    $Y \setminus (Y \setminus f(E))$ is closed and we see that
    $$Y \setminus (Y \setminus f(E))
    = (f(E) \cap Y) \cup (Y \setminus Y)
    = f(E) \cup \emptyset = f(E)$$
    therefore $f$ is closed.
    \item [$(c) \Rightarrow (a)$] Let $f$ be closed. Let $U \subseteq X$ be a
    open set then $X \setminus U$ is closed hence
    $f(X \setminus U) = Y \setminus f(U)$ is also closed since $f$ is closed
    hence $Y \setminus (Y \setminus f(U)) = f(U)$ is an open set. But also we
    know that $(f^{-1})^{-1}(U) = f(U)$ since $f$ is bijective. Therefore
    $f^{-1}$ is also continuous which implies that $f$ is a homeomorphism.   
\end{itemize}
\end{proof}
\cleardoublepage
\begin{proof}{\textbf{Exercise 2.32.}}
\begin{itemize}
    \item [(a)] Let $f: X \to Y$ be a homeomorphism and let
    $x \in X$ with a neighborhood $U \subseteq X$ which is open by definition.
    Since $f$ is a homeomorphism then $f^{-1}$ is continuous hence
    $(f^{-1})^{-1}(U)$ is an open set of $Y$ but $(f^{-1})^{-1}(U) = f(U)$
    because $f$ is bijective therefore $f(U)$ is an open set of $Y$.

    Let $f|_U: U \to f(U)$ be the restriction of $f$ to $U$. Since $f$ is a
    homeomorphism then $f$ is continuous and we know from Exercise 2.18 (c)
    that then $f|_U$ is continuous. Let also $f|_U^{-1}:f(U) \to U$ we know
    $f^{-1}$ is continuous since $f$ is a homeomorphism then because of
    Exercise 2.18 (c) we have that $f|_U^{-1}$ is also continuous.
    Finally, since $f$ is biyective then $f|_U$ is also biyective.
    Therefore $f|_U$ is also a homeomorphism and $f$ is a local homeomorphism.

    \item [(b)] Let $f$ be a local homeomorphism and let $U \subseteq X$ be an
    open subset then for every $x \in U$ there is a neighborhood
    $V_x \subseteq X$ such that $f|_{V_x}$ is a homeomorphism hence
    $(f|_{V_x})^{-1}$ is continuous which implies that $f|_{V_x}(U)$ is an open
    set in $f(V_x)$ and by definition we have that
    $$f|_{V_x}(U) = f(V_x) \cap f(U)$$
    Hence $f|_{V_x}(U)$ is contained in $f(U)$ therefore every
    element of $f(U)$ has a neighborhood $f|_{V_x}(U)$ which is
    contained in $f(U)$ which implies that $f(U)$ is open and $f$
    is an open map.

    Let $B \subseteq Y$ be an open set then for every
    $x \in f^{-1}(B) \subseteq X$ there is a neighborhood $U_x \subseteq X$
    such that $f|_{U_x}$ is a homeomorphism hence $f|_{U_x}$ is continuous
    then $(f|_{U_x})^{-1}(B)$ must be open in $U_x$ so by definition we have that
    $$(f|_{U_x})^{-1}(B) = U_x \cap f^{-1}(B)$$
    Hence $(f|_{U_x})^{-1}(B)$ is contained in $f^{-1}(B)$ therefore every
    element of $f^{-1}(B)$ has a neighborhood $(f|_{U_x})^{-1}(B)$ which is
    contained in $f^{-1}(B)$ which implies that $f^{-1}(B)$ is open and $f$
    is continuous.

    \item [(c)] Let $f$ be a bijective local homeomorphism then $f$ is
    continuous and open from (b). Let $U \subseteq X$ be an open set then
    $f(U) = (f^{-1})^{-1}(U)$ since $f$ is bijective and it's open since $f$
    is an open map. Therefore $f^{-1}$ is also continuous which implies that
    $f$ is a homeomorphism.
\end{itemize}
\end{proof}
\cleardoublepage
\begin{proof}{\textbf{Exercise 2.33.}}
    Let $(y_i) \subseteq Y$ be a sequence that converges to $y \in Y$
    this implies that for every
    neighborhood $U$ of $y$ there exists $N \in \N$ such that $y_i \in U$ for
    all $i \geq N$ but we are considering a trivial topology of $Y$
    then the only possible neighborhood for $y$ is $Y$ hence there is always
    $1 \in \N$ such that for all $i \geq 1$ we have that $y_i \in Y$.
    Therefore since $y$ is arbitrary every sequence $(y_i)$ converges to every
    element in $Y$.
\end{proof}
\begin{proof}{\textbf{Exercise 2.35.}}
    Let $p,q \in X$ then there is $f:X \to \R$ such that $f(p) = 0$ but then
    $f(q) \neq 0$ and let us call $f(q) = r \in \R$.
    We can take an open set $U = (-r/2 , r/2)$ if $r > 0$ or
    $U = (r/2, -r/2)$ otherwise such that $0 \in U$
    also, we can take a different set $V = (r/2, \infty)$ if $r > 0$ and
    $V = (-\infty, r/2)$ otherwise such that $r \in V$.
    We see that $U$ and $V$ are both open sets in $\R$ hence since $f$ is
    continuous we have that $f^{-1}(U)$ and $f^{-1}(V)$ are open disjoint sets
    where $p \in f^{-1}(U)$ and $q \in f^{-1}(V)$ therefore $X$ must be a
    Hausdorff space.
\end{proof}
\begin{proof}{\textbf{Exercise 2.38.}}
    Let us define $\Topo$ to be a topology on a finite set $X$ such that
    $(X, \Topo)$ is a Hausdorff space, we want to show that
    $\Topo = \mathcal{P}(X)$ i.e. $\Topo$ is the discrete topology.
    Let $x,y \in X$ then there is a neighborhood $U_y \in \Topo$ of $x$
    such that $y \not\in U_y$ so we can build $U = \bigcap_{y\neq x} U_y$
    where $U$ is a finite intersection of open
    neighborhoods hence it's an open set where $x \in U$ but $y \not\in U$
    for every $y \in X$ therefore the singletons $\{x\}$ must be open sets
    of $\Topo$. Also, we know that any union of arbitrarily many open
    subsets of $X$ is an open subset of $X$ hence they are in the topology
    $\Topo$ so this implies that it must happen that $\Topo = \mathcal{P}(X)$
    therefore $\mathcal{P}(X)$ is the only topology on a finite set $X$ such
    that $(X, \Topo)$ is a Hausdorff space.
\end{proof}
\begin{proof}{\textbf{Exercise 2.40.}}

    ($\Rightarrow$) Let $U \subseteq X$ be an open subset since $\mathcal{B}$
    is a basis for $X$'s topology then $U$ can be written as
    $U = \bigcup_{\alpha} B_\alpha$ where each $B_\alpha \in \mathcal{B}$
    hence for each $p \in U$ must happen that 
    $p \in \bigcup_{\alpha} B_\alpha$ i.e. $p$ is at least in one
    $B_\alpha \subseteq U$.

    ($\Leftarrow$) If for each $p \in U$ there exist $B \in \mathcal{B}$
    such that $p \in B \subseteq U$ it must happen that
    $U = \bigcup_{\alpha} B_\alpha$ where $B_\alpha \in \mathcal{B}$
    which implies that $U$ is the union of some collection of elements of
    $\mathcal{B}$ therefore $U$ must be an open set.

\end{proof}
\cleardoublepage
\begin{proof}{\textbf{Exercise 2.42.}}
\begin{itemize}
    \item [(a)] Let us consider a topology with basis $\mathcal{B}_\infty$
    for $\R^n$ where each $B \in \mathcal{B}_\infty$ is defined as
    $B_{s/2}(x) = \{y \in \R: d_\infty(x,y) < s/2\}$ where
    $d_\infty(x,y) = \max_{1 \leq i \leq n}|x_i - y_i|$
    we see that each $B_{s/2}(x)$ is an open cube in $\R^n$ of side length $s$.
    On the other hand, we know because of Exercise 2.4(c) that
    $d_{\infty}$ generate the same topology as the Euclidean metric
    i.e. we can generate every open set of the Euclidean topology with
    an arbitrary union of elements of $\mathcal{B}_\infty$. Hence,
    since $\mathcal{B}_1$ is also the collection of open cubes of side length
    $s$ in $\R^{n}$ it must also be a basis for $\R^{n}$.

    \item [(b)] Let $B_{\epsilon}(t)$ be a ball around $t \in \R^n$.
    Let us take now some $p \in B_{\epsilon}(t)$,
    since $\Q$ is dense on $\R$ then for every 
    coordinate $p_i$ we have that there is $x_i, r_i \in \Q$ such that
    $$t_i - \epsilon < r_i < x_i < p_i < x_i + r_i < t_i + \epsilon$$
    This implies that for every $p \in B_{\epsilon}(t)$ there is a ball
    $B_{r}(x)$ such that $ p \in B_{r}(x) \subseteq B_{\epsilon}(t)$.
    Therefore every open set in $\R^n$ can be built as an arbitrary union
    of elements (balls) from $\mathcal{B}_2$ hence $\mathcal{B}_2$
    is a basis for the Euclidean topology on $\R^n$.
\end{itemize}
\end{proof}
\cleardoublepage
\begin{proof}{\textbf{Exercise 2.45.}}
Let $\mathcal{B}$ be a basis for some topology on $X$ we want to prove that
the basis $\mathcal{B}$ satisfies the two properties $(i)$ and $(ii)$.
\begin{itemize}
    \item [(i)] By definition of basis we know that every open subset
    of $X$ is the union of some collection of elements of $\mathcal{B}$.
    Also, we know that $X$ is open so it must happen that
    $\bigcup_{B \in \mathcal{B}} B = X$.

    \item [(ii)] Let $B_1,B_2 \in \mathcal{B}$ and $x \in B_1 \cap B_2$ we want
    to prove there is an element $B_3 \in \mathcal{B}$ such that
    $x \in B_3 \subseteq B_1 \cap B_2$.
    % Let us suppose the opposite i.e. there is no $B_3 \in  \mathcal{B}$
    % such that $x \in B_3 \subseteq B_1 \cap B_2$ we want to arrive at a
    % contradiction.
    Let us take $U = B_1 \cap B_2$ we see that $U$ must be an open set since
    $B_1$ and $B_2$ are also open, then since $\mathcal{B}$ is a basis
    this implies that $U$ must be an arbitrary union of elements of
    $\mathcal{B}$ hence there must be at least
    one set $B \in \mathcal{B}$ such that $x \in B \subseteq U$.
    Therefore we have that $x \in B \subseteq U \subseteq B_1 \cap B_2$
    if we take $B_3 = B$ we are done.
\end{itemize}
\end{proof}
\begin{proof}{\textbf{Exercise 2.51.}}
\begin{itemize}
    \item [(b)] Let $\mathcal{B}$ be a countable basis and let us take
    an element $x_1 \in B_1 \in \mathcal{B}$ then we can build a set
    $A = \{x_i : x_i \in B_i \in \mathcal{B} \text{ for } i \in \N\}$
    we see that $A$ is countable. We want to prove that $A$ is dense in $X$.

    Let us take an open set $U \subseteq X$ since we have a basis $\mathcal{B}$
    we know that $U = \bigcup_\alpha B_\alpha$ then at least one element of $A$
    is in $U$. Therefore $A$ is dense in $X$.
\end{itemize}
\end{proof}
\cleardoublepage
\begin{proof}{\textbf{Exercise 2.54.}}
    Let $M$ be a $0$-manifold, let $p \in M$ and let $U \subseteq M$
    be a neighborhood of $p$ such that it is homeomorphic to
    an open ball in $\R^0$ but an open ball in $\R^0$ is the only point in
    $\R^0$ and since homeomorphism means bijection it must happen that
    $U = \{p\}$. Also, since $M$ is second countable we know there is a
    countable basis $\mathcal{B}$ and hence there is $B \in \mathcal{B}$
    such that $p \in B \subseteq \{p\}$ so must be that $B = \{p\}$
    and we know that $M = \bigcup_{B \in \mathcal{B}} B$ therefore $M$ is 
    a countable discrete space.

    Let now $M$ be a countable discrete space we want to prove it's a
    $0$-manifold.

    Let $p, q \in M$ then there are two neighborhoods $\{p\}$ and $\{q\}$
    such that $\{p\} \cap \{q\} = \emptyset$. Therefore $M$ is Hausdorff.

    Let $\mathcal{B} = \{\{p\}: p \in M\}$ then any open subset of $M$ is the
    union of some collection of elements of $\mathcal{B}$. Therefore $M$ is
    second countable.

    Let $p \in M$, let $0 \in \R^0$ be the only point in $\R^0$, and
    let us define a map $T: \{p\} \to \{0\}$.
    We see that $T$ is a bijection.
    The only topology for $\R^0$ is $\Topo = \{\emptyset, \{0\}\}$ and we see
    that $T^{-1}(\{0\}) = \{p\}$ which is open in $M$ so $T$ is continuous.
    Since $T(\{p\}) = \{0\}$ we have that $T^{-1}$ is also continuous.
    Therefore every point of $M$ has a neighborhood homeomorphic to an open
    ball in $\R^0$ and to $\R^0$ itself. This implies that $M$ is locally
    Euclidean of dimension $0$.

    Finally, since $M$ is Hausdorff, second countable and locally Euclidean of
    dimension $0$ we have that $M$ is a $0$-manifold.
\end{proof}
\end{document}