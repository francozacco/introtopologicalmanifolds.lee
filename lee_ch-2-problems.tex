\documentclass[11pt]{article}
\usepackage{amssymb}
\usepackage{amsthm}
\usepackage{enumitem}
\usepackage{amsmath}
\usepackage{bm}
\usepackage{adjustbox}
\usepackage{mathrsfs}
\usepackage{graphicx}
\usepackage{siunitx}
\usepackage[mathscr]{euscript}


\title{\textbf{Solutions to selected problems on Introduction to Topological Manifolds - John M. Lee.}}
\author{Franco Zacco}
\date{}

\addtolength{\topmargin}{-3cm}
\addtolength{\textheight}{3cm}

\newcommand{\N}{\mathbb{N}}
\newcommand{\Z}{\mathbb{Z}}
\newcommand{\Q}{\mathbb{Q}}
\newcommand{\R}{\mathbb{R}}
\newcommand{\HH}{\mathbb{H}}
\newcommand{\diam}{\text{diam}}
\newcommand{\cl}{\text{cl}}
\newcommand{\bdry}{\text{bdry}}
\newcommand{\inter}{\text{Int}}
\newcommand{\ext}{\text{Ext}}
\newcommand{\Pow}{\mathcal{P}}
\newcommand{\Topo}{\mathcal{T}}
\newcommand{\Or}{\text{ or }}
\newcommand{\setmin}{\setminus}


\theoremstyle{definition}
\newtheorem*{solution*}{Solution}

\begin{document}
\maketitle
\thispagestyle{empty}

\section*{Chapter 2 - Topological Spaces}

\subsection*{Problems}

\begin{proof}{\textbf{2-1}}
    \begin{itemize}
    \item [(a)] We want to show that
    $\Topo_1 = \{U \subseteq X: U = \emptyset \text{ or }X \setminus U \text{ is finite}\}$
    is a topology on $X$.
        \begin{itemize}
            \item [(i)] By definition $\emptyset$ is in $\Topo_1$. If $U = X$
            then $X \setminus X = \emptyset$ and $\emptyset$ is finite then
            $X \in \Topo_1$.
            \item [(ii)] Let $U_1, ..., U_n$ be elements of $\Topo_1$ such that
            $U_i = \emptyset$ or $X \setminus U_i$ is finite for every
            $i$. Also, we see that 
            $$X \setminus (U_1 \cap ... \cap U_n)
            = (X \setminus U_1) \cup ... \cup (X \setminus U_n)$$
            And the finite union of finite sets is 
            itself a finite set hence $U_1 \cap ... \cap U_n \in \Topo_1$.
            We assumed that not all of the elements are empty,  but otherwise
            we already saw that $\emptyset \in \Topo_1$.
            \item [(iii)] Let $(U_\alpha)_{\alpha \in A}$ be a family of
            elements of $\Topo_1$  such that
            $U_i = \emptyset$ or $X \setminus U_i$ is finite for every
            $i$. Also, we have that
            $$X \setminus \bigcup_{\alpha \in A} U_\alpha
            = \bigcap_{\alpha \in A} X \setminus U_\alpha$$
            So this is the intersection between finite sets then itself it's
            a finite set hence $\bigcup_{\alpha \in A} U_\alpha \in \Topo_1$.
        \end{itemize} 
        Therefore $\Topo_1$ is a topology on $X$.
        \item [(b)] We want to show that
        $\Topo_2 = \{U \subseteq X: U = \emptyset \text{ or }X \setminus U \text{ is countable}\}$
        is a topology on $X$.
            \begin{itemize}
                \item [(i)] By definition $\emptyset$ is in $\Topo_2$. If $U = X$
                then $X \setminus X = \emptyset$ and $\emptyset$ is countable then
                $X \in \Topo_2$.
                \item [(ii)] Let $U_1, ..., U_n$ be elements of $\Topo_2$ such that
                $U_i = \emptyset$ or $X \setminus U_i$ is countable for every
                $i$. Also, we see that 
                $$X \setminus (U_1 \cap ... \cap U_n)
                = (X \setminus U_1) \cup ... \cup (X \setminus U_n)$$
                And the finite union of countable sets is 
                itself a countable set hence $U_1 \cap ... \cap U_n \in \Topo_2$.
                We assumed that not all of the elements are empty,  but otherwise
                we already saw that $\emptyset \in \Topo_2$.
                \item [(iii)] Let $(U_\alpha)_{\alpha \in A}$ be a family of
                elements of $\Topo_2$  such that
                $U_i = \emptyset$ or $X \setminus U_i$ is countable for every
                $i$. Also, we have that
                $$X \setminus \bigcup_{\alpha \in A} U_\alpha
                = \bigcap_{\alpha \in A} X \setminus U_\alpha$$
                So this is the intersection between countable sets then itself
                it's a countable set hence
                $\bigcup_{\alpha \in A} U_\alpha \in \Topo_2$.
            \end{itemize} 
            Therefore $\Topo_2$ is a topology on $X$.    
        \item [(c)] We want to show that
        $\Topo_3 = \{U \subseteq X: U = \emptyset \text{ or }p \in U\}$
        is a topology on $X$.
            \begin{itemize}
                \item [(i)] By definition $\emptyset$ is in $\Topo_3$.
                Since $p \in X$ by definition then $X \in \Topo_3$.
                \item [(ii)] Let $U_1, ..., U_n$ be elements of $\Topo_3$ such that
                $U_i = \emptyset$ or $p \in U_i$ for every $i$ then 
                $U_1 \cap ... \cap U_n$ at least have the element $p$ in
                common so $U_1 \cap ... \cap U_n \in \Topo_3$.
                This result is true assuming not every $U_i = \emptyset$
                otherwise $U_1 \cap ... \cap U_n = \emptyset$ and we saw
                that $\emptyset \in \Topo_3$ so anyway 
                $U_1 \cap ... \cap U_n \in \Topo_3$.
                \item [(iii)] Let $(U_\alpha)_{\alpha \in A}$ be a family of
                elements of $\Topo_3$  such that
                $U_i = \emptyset$ or $p \in U_i$ for every $i$. Then
                assuming not every $U_i = \emptyset$ we have that
                $p \in \bigcup_{\alpha \in A} U_\alpha$ which implies that
                $\bigcup_{\alpha \in A} U_\alpha \in \Topo_3$. If every
                $U_i = \emptyset$ then $\bigcup_{\alpha \in A} U_\alpha = \emptyset$
                and we saw that $\emptyset \in \Topo_3$ so anyway 
                $\bigcup_{\alpha \in A} U_\alpha \in \Topo_3$.
            \end{itemize} 
        Therefore $\Topo_3$ is a topology on $X$.        
        \item [(d)] We want to show that
        $\Topo_4 = \{U \subseteq X: U = X \text{ or }p \not\in U\}$
        is a topology on $X$.
            \begin{itemize}
                \item [(i)] By definition $X$ is in $\Topo_4$.
                Since $p \not\in \emptyset$ then $\emptyset \in \Topo_4$.
                \item [(ii)] Let $U_1, ..., U_n$ be elements of $\Topo_4$
                such that $U_i = X$ or $p \not\in U_i$ for every $i$ then 
                $p$ is not in $U_1 \cap ... \cap U_n$.
                This result is true assuming not every $U_i = X$
                otherwise $U_1 \cap ... \cap U_n = X$ and we saw
                that $X \in \Topo_4$ so anyway 
                $U_1 \cap ... \cap U_n \in \Topo_4$.
                \item [(iii)] Let $(U_\alpha)_{\alpha \in A}$ be a family of
                elements of $\Topo_4$  such that
                $U_i = X$ or $p \not\in U_i$ for every $i$. Then
                assuming no $U_i = X$ we have that
                $p \not\in \bigcup_{\alpha \in A} U_\alpha$ which implies that
                $\bigcup_{\alpha \in A} U_\alpha \in \Topo_4$. If some
                $U_i = X$ then $\bigcup_{\alpha \in A} U_\alpha = X$
                and we saw that $X \in \Topo_4$ so anyway 
                $\bigcup_{\alpha \in A} U_\alpha \in \Topo_4$.
            \end{itemize} 
        Therefore $\Topo_4$ is a topology on $X$.
        \item [(e)] We want to determine if
        $\Topo_5 = \{U \subseteq X: U = X \Or X \setminus U \text{ is infinite}\}$
        is a topology on $X$.
        If we let $X = \Z$ then $\Z^+ \in \Topo_5$ and $\Z^- \in \Topo_5$ but
        $\Z \setmin (\Z^+ \cup \Z^-) = \{0\}$ but $\{0\}$ is finite so
        $(\Z^+ \cup \Z^-) \not\in \Topo_5$. Therefore $\Topo_5$ is not 
        a topology on $X$.
    \end{itemize}
\end{proof}
\cleardoublepage
\begin{proof}{\textbf{2-3}}
    \begin{itemize}
    \item [(a)]
    Let $x \in \overline{X \setmin B}$ then for every neighborhood $U$ where
    $x \in U$ contains a point of $X \setmin B$ this implies that no
    neighborhood that contains $x$ is in $\inter(B)$ hence 
    $x \in X \setmin \inter(B)$ and
    $\overline{X \setmin B} \subseteq X \setmin \inter(B)$.

    Let $x \in X \setmin \inter(B)$ then $x \not\in \inter(B)$ so there is no 
    neighborhood of $x$ contained in $B$ then for every neighborhood $U$ that
    contains $x$ we have that $U \cap X \setmin B \neq \emptyset$ hence
    $x \in \overline{X \setmin B}$ and
    $X \setmin \inter(B) \subseteq \overline{X \setmin B}$.

    Therefore $\overline{X \setmin B} = X \setmin \inter(B)$.

    \item [(b)]
    Let $ x \in \inter(X \setmin B)$ then $x$ has a neighborhood contained in
    $X \setmin B$ but then $x$ is also in $\ext(B) = X \setmin \overline{B}$
    then $\inter(X \setmin B) \subseteq X \setmin \overline{B}$.

    In the same way if $x \in X \setmin \overline{B} = \ext(B)$ then it has a
    neighborhood in $X \setmin B$ hence $X \setmin B$ is open and hence
    $x \in \inter(X \setmin B)$ this implies that
    $ X \setmin \overline{B} \subseteq \inter(X \setmin B)$.
    
    Therefore $\inter(X \setmin B) = X \setmin \overline{B}$.
    \end{itemize}
\end{proof}
\cleardoublepage
\begin{proof}{\textbf{2-4}}
    \begin{itemize}
    \item [(a)] We know that $\bigcap_{A \in \mathcal{A}} \overline{A}$ is 
    a closed set where
    $\bigcap_{A \in \mathcal{A}} A \subseteq \bigcap_{A \in \mathcal{A}} \overline{A}$
    but also we know that the closure of $\bigcap_{A \in \mathcal{A}} A$ is the
    smallest closed set containing $\bigcap_{A \in \mathcal{A}} A$. Therefore
    it must happen that
    $\overline{\bigcap_{A \in \mathcal{A}} A} \subseteq \bigcap_{A \in \mathcal{A}} \overline{A}$.

    Let $\mathcal{A} = \{(0,1), (1, 2)\}$ where they are intervals of $\R$ then
    $\bigcap_{A \in \mathcal{A}} \overline{A} = \{1\}$ and
    $\overline{\bigcap_{A \in \mathcal{A}} A} = \emptyset$ so we see that
    $\bigcap_{A \in \mathcal{A}} \overline{A}
    \not\subseteq \overline{\bigcap_{A \in \mathcal{A}} A}$. Therefore when
    $\mathcal{A}$ is a finite collection the equality is not preserved.

    \item [(b)] We know that $A \subseteq \bigcup_{A \in \mathcal{A}} A$ then
    $A \subseteq \overline{\bigcup_{A \in \mathcal{A}} A}$ also
    $\overline{A} \subseteq \overline{\bigcup_{A \in \mathcal{A}} A}$ therefore
    $$\bigcup_{A \in \mathcal{A}} \overline{A}
    \subseteq \overline{\bigcup_{A \in \mathcal{A}} A}$$

    Let us assume now that $\mathcal{A}$ is a finite collection. Then
    we see that $\bigcup_{A \in \mathcal{A}} \overline{A}$ is a closed set 
    and that
    $\bigcup_{A \in \mathcal{A}} A
    \subseteq \bigcup_{A \in \mathcal{A}} \overline{A}$
    but also we know that $\overline{\bigcup_{A \in \mathcal{A}} A}$ is the
    smallest closed set that contains $\bigcup_{A \in \mathcal{A}} A$ therefore
    it must happen that 
    $$\overline{\bigcup_{A \in \mathcal{A}} A}
    \subseteq \bigcup_{A \in \mathcal{A}} \overline{A}$$
    Hence the equality holds when $\mathcal{A}$ is a finite collection.

    \item [(c)] Let $x \in \inter(\bigcap_{A \in \mathcal{A}} A)$ then there is
    a neighborhood such that $U \subseteq \bigcap_{A \in \mathcal{A}} A$ hence
    $U \subseteq A$ for every $A \in \bigcap_{A \in \mathcal{A}} A$ hence
    because of Proposition 2.8 (a) we have that $x \in \inter(A)$ but since
    this is true for every $A$ then must be that 
    $x \in \bigcap_{A \in \mathcal{A}} \inter(A)$. Therefore
    $$\inter\bigg(\bigcap_{A \in \mathcal{A}} A\bigg)
    \subseteq \bigcap_{A \in \mathcal{A}} \inter(A)$$

    Let us assume now that $\mathcal{A}$ is a finite collection. Then we see
    that $\bigcap_{A \in \mathcal{A}} \inter(A)$ is an open set and
    $\bigcap_{A \in \mathcal{A}} \inter(A) \subseteq \bigcap_{A \in \mathcal{A}} A$
    but also we know that $\inter(\bigcap_{A \in \mathcal{A}} A)$ is the
    biggest open set contained in $\bigcap_{A \in \mathcal{A}} A$ therefore
    it must be that 
    $$\bigcap_{A \in \mathcal{A}} \inter(A)
    \subseteq \inter\bigg(\bigcap_{A \in \mathcal{A}} A\bigg)$$
    Hence the equality holds when $\mathcal{A}$ is a finite collection.
\cleardoublepage
    \item [(d)] We know that $\bigcup_{A \in \mathcal{A}} \inter(A)$ is open
    since it's an arbitrary union of open sets and also we see that
    $\bigcup_{A \in \mathcal{A}} \inter(A)
    \subseteq  \bigcup_{A \in \mathcal{A}} A$ but also we know that
    $\inter\bigg(\bigcup_{A \in \mathcal{A}} A\bigg)$ is the biggest open set
    contained in $\bigcup_{A \in \mathcal{A}} A$ therefore it must happen that
    $$\bigcup_{A \in \mathcal{A}} \inter(A)
    \subseteq \inter\bigg(\bigcup_{A \in \mathcal{A}} A\bigg)$$

    Let $\mathcal{A} = \{[0,1], [1, 2]\}$ where they are intervals of $\R$ then
    $\bigcup_{A \in \mathcal{A}} \inter(A) = (0,1) \cup (1,2)$ and
    $\inter(\bigcup_{A \in \mathcal{A}} A) = (0,2)$ so we see that
    $\inter(\bigcup_{A \in \mathcal{A}} A)
    \not\subseteq \bigcup_{A \in \mathcal{A}} \inter(A)$. Therefore when
    $\mathcal{A}$ is a finite collection the equality is not preserved.
    \end{itemize}
\end{proof}
\begin{proof}{\textbf{2-5}}
\begin{itemize}
    \item [(a)] Let $X = [0, \infty) \times \{0\} \subset \R^2$ and 
    $Y = [-1, 1] \times \{0\} \subset \R^2$ such that
    $f(x) = \sin(1/x)$ for $x > 0$ and if $x = 0$ we have that $f(0) = 0$.
    
    Let $U \subset X$ be an open set such that $U$ is of the form $(a,b)$
    which is a basis and
    since $U\subset (0, \infty)$ and $f$ is continuous in $(0, \infty)$ because
    of the intermediate value theorem, we have that $f(U)$ is also open,
    therefore $f$ is an open map. 

    Let $E = \{1/n: n \in \N\} \cup \{0\}$ we know that $E \subset X$ and $E$
    is a closed set then let us consider
    $f(E) = \{\sin(n): n\in \N\} \cup \{0\}$ we see that $f(E)$ is a dense set
    hence if we let $x \in [-1,1]$ such that $x \not\in f(E)$ then every
    neighborhood of $x$ will have a point of $f(E)$ so $x$ is a limit point
    of $f(E)$ which is not contained in $f(E)$, therefore $f(E)$ is not
    closed and $f$ is not a closed map.

    Let us consider now a sequence $(x_n) \subseteq X$ defined as
    $x_n = 1/(\pi n + \pi/2)$ we see that
    $x_n \to 0$ but $f(x_n) = -1$ if $n$ is odd and $f(x_n) = 1$ if $n$
    is even hence $f$ does not converge and $f$ is therefore not continuous.

    \item [(b)] Let $X = \R\times\{0\}$ and $Y = \R\times \{0\}$ such
    that $f(x) = 0$ if $x \neq 0$ and $f(x) = 1$ if $x = 0$.

    Let $E \subset X$ be a closed set then $f(E) = \{0\}$ or $f(E) = \{1\}$
    or $f(E) = \{0, 1\}$ in any case they are closed sets since they are
    singletons hence $f$ is a closed map.

    Let $U \subset X$ be an open set then $f(U) \subseteq \{0,1\}$ and we know
    that neither $\{0,1\}$ nor $\{0\}$ nor $\{1\}$ are open sets hence
    $f$ is not an open map.

    Finally $f$ is not continuous at $y=0$ since if we take $\epsilon = 1/2$ we have
    that $|f(x) - f(0)| = |0 - 1| = 1 > 1/2$ no matter which $\delta > 0$
    we take.

    \item [(c)] Let $X = \R \times \{0\}$ and $Y = \R \times \{0\}$
    such that $f(x) = 0$ if $x \leq 0$ and $f(x) = \arctan(x)$ if $x > 0$.
    We see that $f$ is continuous.

    Let $U = (-1,0) \subset \R$ we know $U$ is an open set of $X$ then
    $f(U) =\{0\}$  but $\{0\}$ is not an open set in $Y = \R$. Therefore $f$
    is not an open map.

    Let $E = [0, \infty) \subset X$ we know that $E$ is closed in $X$, also we
    see that $f(E) = [0,1)$ but $[0,1)$ is not an open nor a closed set in
    $Y = \R$ therefore $f$ is not a closed map.

    \item [(d)] Let $X = [0,\infty) \times \{0\}$ and $Y = \R \times \{0\}$
    such that $f(x) = \arctan(x)$ we see that $f$ is continuous.

    Let $U \subset X$ be an open set such that $U$ is of the form $(a,b)$
    which is a basis and
    since $U\subset (0, \infty)$ and $f$ is continuous in $(0, \infty)$ because
    of the intermediate value theorem, we have that $f(U)$ is also open,
    therefore $f$ is an open map. 

    Let $E = [0, \infty) \subset X$ we know that $E$ is closed in $X$, also we
    see that $f(E) = [0,1)$ but $[0,1)$ is not an open nor a closed set in
    $Y = \R$ therefore $f$ is not a closed map.

    \item [(e)] Let $X = \R\times\{0\}$ and $Y = \R\times \{0\}$ such
    that $f(x) = 0$ we see that $f$ is continuous.

    Let $E \subset X$ be a closed set then $f(E) = \{0\}$ which is a closed
    set in $Y = \R$ since it is a singleton hence $f$ is a closed map.

    Let $U \subset X$ be an open set then $f(U) = \{0\}$ and we know
    that $\{0\}$ is not an open sets in $Y = \R$ hence $f$ is not an open map.

\cleardoublepage
    \item [(f)] Let $X = \R\times\{0\}$ and
    $Y = (-\infty, 1] \cup (2, \infty)\times \{0\}$ such
    that $f(x) = x$ if $x \leq 1$ and $f(x) = 2x$ if $x > 1$.

    Let $U \subset X$ be an open set such that $U$ is of the form $(a,b)$ which
    is a basis hence it's valid for any open set if
    $(a,b) \subset (-\infty, 1]$ we know that $f$ is continuous
    in this interval so because of the Intermediate Value Theorem $f(U)$
    is an open map and in the same way if
    $(a,b) \subset (1, \infty)$ then $f(U)$ is open because $f$ is continuous
    in this interval and the Intermediate Value Theorem.
    Now suppose $(a,b)$ such that $a < 1 < b$ then
    $(a,b) = (a, 1] \cup (1, b)$ and we see that $f((a, 1]) = (a, 1]$ which is
    open in $(-\infty, 1]$ since $(-\infty, 1] \setminus (a, 1] = (-\infty, a]$
    which we know is a closed set hence $(a, 1]$ is an open set.
    Also, we have that $f((1, b)) = (2, 2b)$ which is an open set so in this
    case $f((a,b))$ is the union of two open sets i.e. it's an open set.
    Therefore $f$ is an open map.

    Let $E \subseteq X$ be a closed set and let us take a sequence
    $(y_n) \subseteq f(E)$ such that $y_n \to y$ we want to prove that
    $y \in f(E)$ which would imply that $f(E)$ is closed.
    By definition there is $x_n \in E$ such that $f(x_n) = y_n$.
    
    But also we know that $x_n = y_n$ or
    $x_n = y_n /2$ or a combination of both but only for a finite number of
    points by the definition of $f$.

    In the first case, this implies that $x_n \to x$ where $x = y$ and since
    $E$ is a closed set then $x \in E$ but also we know that $f$ is continuous
    in $(-\infty, 1]$ hence $y \in f(E)$.
        
    Lastly if $x_n = y_n /2$ we have that
    $x_n \to x/2$ where $x/2 = y$ and since
    $E$ is a closed set then $x \in E$ but also we know that $f$ is continuous
    in $(1,\infty)$ hence $y \in f(E)$.

    Therefore $f$ is a closed map.
\end{itemize}
\end{proof}
\begin{proof}{\textbf{2-6}}
\begin{itemize}
    \item [(a)]
    ($\Rightarrow$) Let $f$ be continuous and let $A \subseteq X$ also
    $\overline{f(A)}$ is closed on $Y$ hence
    $f^{-1}(\overline{f(A)})$ is closed on $X$ because $f$ is
    continuous but also we know that
    $A \subseteq f^{-1}(\overline{f(A)})$ and since $f^{-1}(\overline{f(A)})$
    is closed it must happen that
    $\overline{A} \subseteq f^{-1}(\overline{f(A)})$ this in turn implies that
    $f(\overline{A}) \subseteq f(f^{-1}(\overline{f(A)})) \subseteq \overline{f(A)}$.
    
    ($\Leftarrow$) Let $B \subseteq Y$ be a closed set we want to show that
    $A = f^{-1}(B)$ is also closed in $X$. Given that
    $f(\overline{A}) \subseteq \overline{f(A)}$
    we have that
    \begin{align*}
        f(\overline{A}) \subseteq \overline{f(A)} =
        \overline{f(f^{-1}(B))} \subseteq \overline{B} = B
    \end{align*}
    since $B$ is closed. Then we have that $f(\overline{A}) \subseteq B$ hence
    $$\overline{A} \subseteq  f^{-1}(f(\overline{A})) \subseteq f^{-1}(B) = A$$
    Therefore $A$ is closed since it contains $\overline{A}$.

    \item [(b)]
    ($\Rightarrow$) Let $f$ be a closed map and let $A$ be any set of $X$
    we know that $A \subseteq \overline{A}$ then $f(A)\subseteq f(\overline{A})$
    but since $f$ is a closed map then $f(\overline{A})$ is closed which
    implies that the closure of $f(A)$ must be contained in $f(\overline{A})$
    i.e. $\overline{f(A)} \subseteq f(\overline{A})$.

    ($\Leftarrow$) Let $A$ be a closed set of $X$ we want to prove that $f(A)$
    is also closed. Since $A$ is closed we have that $A = \overline{A}$ hence
    $f(A) = f(\overline{A})$ but we know that
    $\overline{f(A)} \subseteq f(\overline{A}) = f(A)$
    thus the closure of $f(A)$ is contained or equal to $f(A)$ therefore
    $f(A)$ is closed implying that $f$ is a closed map.

    \item [(c)]
    ($\Rightarrow$) Let $f$ be continuous and let $B \subseteq Y$ we know that
    by definition $\inter(B) \subseteq B$ so $f^{-1}(\inter(B)) \subseteq f^{-1}(B)$ 
    we know that $f^{-1}(\inter(B))$ is open since $f$ is continuous and
    $\inter(B)$ is an open set hence it must also happen that
    $f^{-1}(\inter(B)) \subseteq \inter(f^{-1}(B))$ since by definition
    $\inter(f^{-1}(B))$ is the largest open set contained in $f^{-1}(B)$.

    ($\Leftarrow$) Let $B \subseteq Y$ be an open set we want to prove that
    $f^{-1}(B)$ is an open set which implies that $f$ is continuous.
    Since $B$ is open we have that $B = \inter(B)$
    but also we know that $f^{-1}(B) = f^{-1}(\inter(B)) \subseteq \inter(f^{-1}(B))$
    and by definition we know that $\inter(f^{-1}(B)) \subseteq f^{-1}(B)$
    therefore $\inter(f^{-1}(B)) = f^{-1}(B)$ which implies that $f^{-1}(B)$
    is an open set.

    \item [(d)]
    ($\Rightarrow$) Let $f$ be an open map and let $B \subseteq Y$
    also let us name $C = \inter(f^{-1}(B)) \subseteq f^{-1}(B)$
    then $f(C)\subseteq f(f^{-1}(B)) \subseteq B$ where
    $f(C)$ is open since $f$ is an open map hence it must
    also happen that $f(C) \subseteq \inter(B)$ since $\inter(B)$ is the biggest
    open set contained in $B$ so we have that
    $\inter(f^{-1}(B)) = C \subseteq f^{-1}(f(C)) \subseteq f^{-1}(\inter(B))$.

    ($\Leftarrow$) Let $A \subseteq X$ be an open set we want to prove that
    $f(A)$ is an open set too which implies that $f$ is open.
    Since $f(A)$ is a set of $Y$ we know that
    $\inter (f^{-1}(f(A))) \subseteq f^{-1}(\inter(f(A)))$ also we have that
    $\inter A \subseteq \inter (f^{-1}(f(A)))$ then joining this two
    equations and applying $f$ to both sides we have that
    $$f(A) = f(\inter(A)) \subseteq f(f^{-1}(\inter(f(A)))) \subseteq \inter(f(A)) $$
    where we used that $\inter(A) = A$ since $A$ is an open set.
    Therefore we have that $f(A) \subseteq \inter(f(A))$ but also by definition
    we know that $\inter(f(A)) \subseteq f(A)$ which implies that
    $\inter(f(A)) = f(A)$ thus $f(A)$ is an open set and $f$ is an open map.
\end{itemize}
\end{proof}
\cleardoublepage
\begin{proof}{\textbf{2-7}}
    Let $X$ be a Hausdorff space where $A \subseteq X$.
    Let $p \in X$ be a limit point of $A$ and suppose $U$ is a neighborhood
    of $p$ that contains finitely many points of $A$ we want to arrive at a
    contradiction. Suppose $\{p_1, p_2,..., p_n\} \subset A$ is the set of
    points that are in $U$ other than $p$.
    Also, since $X$ is a Hausdorff space we know
    that there is a set $U'$ that contains $p$ but does not contain any point
    from the set $\{p_1, p_2,..., p_n\}$ hence $U' = \{p\}$
    but this is a contradiction since $p$ is a limit point so it must contain
    at least one point of $A$.
    Therefore if $p$ is a limit point of $A$ then every neighborhood of $p$
    must contain infinitely many points of $A$.
\end{proof}
\begin{proof}{\textbf{2-9}}
\begin{itemize}
    \item [(a)] Let a map $f:D \to A$ where $D$ is a discrete space and $A$
    is any arbitrary topological space, we want to prove
    that $f$ is continuous.
    Let us take an open set $U \subseteq A$ then $f^{-1}(U)$ must be a 
    set of $D$ but since $D$ is a discrete space (with the discrete topology)
    it must happen that $f^{-1}(U)$ is an open set of $D$. Therefore every map
    from $D$ to $A$ is continuous.

    \item [(b)] Let a map $f: A \to T$ where $T$ is a space with the trivial
    topology and $A$ is any arbitrary topological space, we want to prove
    that $f$ is continuous.\\
    Let us take an open set $U \subseteq T$ then it must happen that $U = T$
    or $U = \emptyset$ since $T$ is a space with the trivial topology
    then $f^{-1}(U) = f^{-1}(T) = A$ or
    $f^{-1}(U) = f^{-1}(\emptyset) = \emptyset$ where both $A$ and $\emptyset$
    are open sets by definition. Therefore every map from $A$ to $T$ is
    continuous.

    \item [(c)] Let us suppose that a map $f: T \to H$ where $T$ is a space
    with the trivial topology and $H$ is a Hausdorff space
    is continuous and it's not a constant map, we want to arrive
    at a contradiction.\\
    Let $x,y \in T$ such that $f(x) \neq f(y)$ since $H$ is a Hausdorff space
    it must happen that there is $f(x) \in U_x \subseteq H$ and
    $f(y) \in U_y \subseteq H$ such that $U_x \cap U_y = \emptyset$.
    On the other hand, we know that $f$ is continuous then $f^{-1}(U_x)$
    must be an open set where either $f^{-1}(U_x) = T$ or
    $f^{-1}(U_x) = \emptyset$ and since $x \in f^{-1}(U_x)$ it must happen
    that $f^{-1}(U_x) = T$ because of the same reason we have that
    $y \in f^{-1}(U_y) = T$ hence $x \in f^{-1}(U_y)$ but then $f(x) \in U_y$,
    a contradiction.
    Therefore the only continuous maps from $T$ to $H$ are the constant maps.
\end{itemize}
\end{proof}
\cleardoublepage
\begin{proof}{\textbf{2-10}}
    Let $f,g: X \to Y$ be two continuous maps and $Y$ a Hausdorff space.
    We want to prove that $U = \{x \in X: f(x) = g(x)\}$ is closed on $X$.
    Let us see that $X \setmin U = \{x \in X: f(x) \neq g(x)\}$ and let us
    take $x \in X \setmin U$ then since $f(x) \neq g(x)$ and $Y$ is a Hausdorff
    space there must be $V_{f(x)}, V_{g(x)} \subseteq Y$ such that
    $V_{f(x)} \cap V_{g(x)} = \emptyset$.\\
    We want to prove now that
    $f^{-1}(V_{f(x)}) \cap g^{-1}(V_{g(x)}) \subset X \setmin U$
    suppose there is $y \in f^{-1}(V_{f(x)}) \cap g^{-1}(V_{g(x)})$ such that
    $y \not\in X \setmin U$ we want to arrive at a contradiction.
    Since $y \in f^{-1}(V_{f(x)})$ we have that
    $f(y) \in V_{f(x)}$ but also by the same reasoning we have that
    $g(y) \in V_{g(x)}$ and since $y \not\in X \setmin U$ it must happen that
    $f(y) = g(y)$ thus $V_{f(x)} \cap V_{g(x)} \neq \emptyset$ which is a
    contradiction. Therefore it must be that
    $f^{-1}(V_{f(x)}) \cap g^{-1}(V_{g(x)}) \subset X \setmin U$.\\
    Finally, since this must be true for every $x \in X \setmin U$ then
    $$X \setmin U
    = \bigcup_{x \in X \setmin U} f^{-1}(V_{f(x)}) \cap g^{-1}(V_{g(x)})$$
    hence $X \setmin U$ is open since it's an arbitrary union of open sets,
    which implies that $U = \{x \in X: f(x) = g(x)\}$ is a closed set.

    Let now $f,g:X \to Y$ be two continuous maps where $X = Y = \R$ with the
    trivial topology such that $f(x)=x$ and $g(x) = 0$ for all $x \in X$.
    We see that $U = \{x \in X: f(x) = g(x)\} = \{0\}$ but $U$ is not closed
    since the only closed sets are $X = \R$ and $\emptyset$.
\end{proof}
\cleardoublepage
\begin{proof}{\textbf{2-14}}
\begin{itemize}
    \item [(a)] ($\Rightarrow$) Let $x \in \overline{A}$ since $X$ is a first
    countable space then there is $\mathcal{B}_x$ a nested neighborhood
    basis for $X$ at $x$ and hence there is a sequence $(U_i)_{i=1}^{\infty}$
    of neighborhoods of $x$ such that $U_{i+1} \subseteq U_i$.
    On the other hand, since $x \in \overline{A}$ we know that every
    neighborhood of $x$ contains a point of $A$ then we can build a sequence
    $(x_i) \subseteq A$ such that $x_i \in U_i$ and $x_i \to x$ as we wanted.
    
    ($\Leftarrow$) Let $(x_n) \subset A$ be a sequence such that $x_n \to x$
    where $x \in X$ we want to show that $x \in \overline{A}$. By
    definition of convergence, for every neighborhood $U$ of $x$ there is
    $N \in \N$ such that $x_n \in U$ for all $n \geq N$ but we know that
    $x_n \in A$ therefore $x \in \overline{A}$.

    \item [(b)] ($\Rightarrow$) Let $x \in \inter A$ and let
    $(x_n) \subseteq X$ be a sequence that converges to $x$ then for every
    neighborhood $U$ of $x$ there is $N \in \N$ such that $x_n \in U$ for all
    $n \geq N$ if we take $U = \inter A$ then $(x_n)$ is be eventually in $A$.

    ($\Leftarrow$) Let $(x_n) \subseteq X$ be a sequence that converges to
    $x \in X$ and is eventually in $A$ we want to prove that $x \in \inter A$.
    Let us note that $\inter A = X \setmin \overline{X \setmin A}$ so if $x$
    is not in $\inter A$ it must be in $\overline{X \setmin A}$, let us suppose
    this is the case, we want to arrive at a contradiction.
    By what we proved in part (a) if $x \in \overline{X \setmin A}$ then it is
    a limit of a sequence of points in $X \setmin A$ hence we have a sequence
    that converges to $x$ but it's not eventually in $A$, a contradiction.
    Therefore it must be that $x \in \inter A$.

    \item [(c)] ($\Rightarrow$) Let $A$ be closed in $X$ and let
    $(x_n) \subseteq A$ be a convergent sequence that converges to $x \in X$.
    We know that $A = \overline{A}$ and because of part (a) since $x$ is 
    the limit of a sequence of points in $A$ must happen that
    $x \in \overline{A} = A$. Therefore $A$ contains every limit of every
    convergent sequence of points in $A$.

    ($\Leftarrow$) Let $x \in \overline{A}$ then because of part (a) $x$ is the
    limit of a sequence $(x_n) \subseteq A$ but we know $A$ contains every
    limit of every convergent sequence of points in $A$ since $(x_n)$ is
    convergent then must happen that $x \in A$ and this happens for every
    point in $\overline{A}$ therefore we have that $ \overline{A} \subseteq A$
    which implies by definition that $A = \overline{A}$ hence $A$ is closed in
    $X$.

    \item [(d)] ($\Rightarrow$) Let $A$ be open in $X$ and let
    $(x_n) \subseteq X$ such that $x_n \to x$ where $x \in A$. We know that
    $A = \inter A$ and because of part (b) since $x \in A = \inter A$ then
    every sequence converging to it is eventually in $A$.
    
    ($\Leftarrow$) Let $x \in A$ and $(x_n)\subseteq X$ such that $x_n \to x$
    we know that $(x_n)$ is eventually in $A$ then by part (b) we know that 
    $x \in \inter A$ so we have that $A \subseteq \inter A$ which implies
    by definition of interior that $A = \inter A$ hence $A$ is open in $X$. 
\end{itemize}
\end{proof}
\begin{proof}{\textbf{2-15}}
\begin{itemize}
    \item [(a)] Let $f:X \to Y$ be a continuous map and $p_n \to p$ in $X$.
    We want to show that $f(p_n) \to f(p)$.

    Let us take a neighborhood $U$ of $f(p)$ in $Y$ then since $f$ is continuous
    we have that $f^{-1}(U) \subseteq X$ is open where $p \in f^{-1}(U)$
    but also we know that $p_n \to p$ so by definition there is $N \in \N$
    such that for all $i \geq N$ we have that $p_i \in f^{-1}(U)$.
    Also, in general, we know that $f(f^{-1}(U)) \subseteq U$ hence
    $f(p_i) \in U$ this implies that for any neighborhood $U \subseteq Y$
    of $f(p)$ we can find an $N \in \N$ such that for all $i \geq N$ we have
    that $f(p_i) \in U$, therefore $f(p_i) \to f(p)$.

    \item [(b)] Let $X$ be first countable and $f: X \to Y$ be a map such that
    $p_n \to p$ in $X$ implies that $f(p_n) \to f(p)$ in $Y$,
    we want to prove that $f$ is continuous.

    Let us suppose $f$ is not continuous we want to arrive at a contradiction.
    If $f$ is not continuous then there is a neighborhood $U \subseteq Y$ of
    $f(p)$ such that  $f^{-1}(U)$ is not open.

    We know that $f(p_n) \to f(p)$ so there is $N \in \N$ such that for all
    $i \geq N$ we have that $f(p_i) \in U$ this implies that
    $p \in f^{-1}(U)$ and $p_i \in f^{-1}(U)$ and since $p_n \to p$
    we see that $(p_n)$ is eventually in $f^{-1}(U)$ but also we know that
    $X$ is first countable hence $f^{-1}(U)$ is open, a contradiction.
    Therefore $f$ must be continuous.
\end{itemize}
\end{proof}
\cleardoublepage
\begin{proof}{\textbf{2-18}}
\begin{itemize}
    \item [(a)] Let us take some $p \in \R$ and let us define the topology of
    $\R$ as
    $$\mathcal{T} = \{U \subseteq \R: U= \emptyset \text{ or } p \in U\}$$
    First, we want to prove that $\R$ with this topology is first countable.
    Let us define a collection $\mathcal{B}$ with only one set $\{p\}$ which
    is open by definition of the topology. We see that every neighborhood $U$
    of $p$ contains $p$.
    This implies that $(\R, \mathcal{T})$ is first countable.

    Now we want to prove $(\R, \mathcal{T})$ is separable. Let us take the
    set $\{p\}$. By definition every open set $U$ contains $p$ so $\{p\}$ is
    dense and countable in $\R$ hence $(\R, \mathcal{T})$ is separable.

    Let us suppose now that $(\R, \mathcal{T})$ is second countable we want to
    arrive at a contradiction. Then $(\R, \mathcal{T})$ admits a countable
    basis $\{U_n\}$. Also, we know that for each $x \in \R$ the set $\{p,x\}$
    is open then there must be some $n$ for which $x \in U_n \subseteq \{p,x\}$
    but since this must be true for each $x \in \R$ and $\R$ is not countable
    then $\{U_n\}$ is uncountable, a contradiction.
    Therefore $(\R, \mathcal{T})$ is not second countable.

    Finally, let us suppose that $(\R, \mathcal{T})$ is Lindelöf we want to
    arrive at a contradiction. Then let us take 
    $\mathcal{U} = \{\{p,x\}: x \in \R\}$ as an open cover of $\R$ since each
    set $\{p,x\}$ is open by definition. We know it must have a countable
    subcover $\{U_n\}$. Let $x \in \R$ then there is $U_n$ such that
    $x \in U_n$ for some $n \in \N$ but by definition $U_n = \{p, x\}$
    and this must be this way for every $x \in \R$ then $\{U_n\}$
    is uncountable which is a contradiction.
    Therefore $(\R, \mathcal{T})$ is not Lindelöf.
    \cleardoublepage
    \item [(b)] Let us take some $p \in \R$ and let us define the topology of
    $\R$ as
    $$\mathcal{T} = \{U \subseteq \R: U= \R \text{ or } p \not\in U\}$$
    First, we want to prove that $\R$ with this topology is first countable.
    Let $x \in \R$ such that $x \neq p$ also let us define a collection with
    one element as $\mathcal{B} = \{\{x\}\}$ where $\{x\}$ is open by
    definition of the topology.
    We see that every neighborhood $U$ of $x$ contains $x$. 
    Now let us consider the case where we take $x = p$ then the only
    neighborhood for $p$ is $\R$ so if we take
    $\mathcal{B} = \{\R\}$ we have a countable neighborhood basis for $p$ too. 
    This implies that $(\R, \mathcal{T})$ is first countable.

    Now we want to prove $(\R, \mathcal{T})$ is Lindelöf.
    Let $\mathcal{U}$ be some open cover of $\R$ then $\mathcal{U}$ is 
    a collection of $U \in \Topo$ but one of them must be $\R$ since otherwise
    $p$ is not covered by definition. 
    Then we can take a subcover $\mathcal{U}' = \{\R\}$ which is countable and
    still covers $\R$. Therefore $(\R, \mathcal{T})$ is Lindelöf.
    
    Let us suppose now that $(\R, \mathcal{T})$ is second countable we want to
    arrive at a contradiction. Then $(\R, \mathcal{T})$ admits a countable
    basis $\{U_n\}$. Also, we know that for each $x \in \R$ such that
    $x \neq p$ the set $\{x\}$ is open then there must be some $n$ for which
    $x \in U_n \subseteq \{x\}$ but since this must be true for each
    $x \in \R$ where $x \neq p$ and $\R \setmin \{p\}$ is not countable then
    $\{U_n\}$ must be uncountable too, a contradiction.
    Therefore $(\R, \mathcal{T})$ is not second countable.

    Finally, let us suppose that $(\R, \mathcal{T})$ is separable we want to
    arrive at a contradiction. Then there must be a set $U$ which is dense
    and countable in $\R$. Let $x \in \R$ such that $x \neq p$. We know that
    the set $\{x\}$ is open by definition then $U$ must contain a point of
    $\{x\}$ hence $\{x\} \subseteq U$ but this
    must happen for every $x \in \R \setmin \{p\}$ which is uncountable then
    $U$ must be uncountable which is a contradiction.
    Therefore $(\R, \mathcal{T})$ is not separable.

    \cleardoublepage
    \item [(c)] Let us define the topology of $\R$ as
    $$\mathcal{T}=
    \{U \subseteq \R: U= \emptyset \text{ or }
    \R \setmin U \text{ is finite}\}$$
    First, we want to prove that $\R$ with this topology is separable.
    Let $D$ be an infinite countable set in $\R$ and let $U \in \Topo$ be an
    open set such that $\R \setmin U$ is finite.
    Let us suppose that no element of $D$ is in $U$ we want to arrive at a
    contradiction. Then it must be in $D \subseteq  \R \setmin U$ but
    $\R \setmin U$ is finite but $D$ is inifinite so we have a contradiction.
    Therefore since this must be true for any $U \in \Topo$ then $D$ is dense.
    
    Now we want to prove $(\R, \mathcal{T})$ is Lindelöf.
    Let $\mathcal{U}$ be some open cover of $\R$ then $\mathcal{U}$ is 
    a collection of sets $U$ from $\Topo$. Let us take any $U_1 \in \mathcal{U}$
    then $\R \setmin U_1$ is a finite set, let us suppose it has $n$ elements
    so there are $n$ points of $\R$ that are not covered by $U_1$ so there are
    at most $n$ sets of $\mathcal{U}$ necessary to cover $\R$ completely,
    hence we can build in the worst-case
    $\mathcal{U}' = \{U_1, U_2,...,U_{n+1}\} \subseteq \mathcal{U}$
    which is a countable subcover of $\mathcal{U}$.
    Therefore $(\R, \mathcal{T})$ is Lindelöf.

    Let us suppose now that $(\R, \mathcal{T})$ is first countable, we want to
    arrive at a contradiction. Then for each $x \in \R$ there is a countable
    neighborhood basis $\{U_n\}$ hence for every neighborhood $U$ of $x$
    there must be some $U_n$ such that $U_n \subseteq U$.
    Let us we take $U' = \bigcap_{i=1}^{\infty} U_n$
    we see that this set must contain $\R$ but countably many points.
    So there is some $y \in U'$ where $y \neq x$ which we can use to build a
    neighborhood of $x$ as $\R \setmin \{y\}$.
    We see that $y \in U_n$ for all $n$ but $y \not\in \R \setmin \{y\}$
    so there is no $U_n$ such that $U_n \subseteq \R \setmin \{y\}$ which
    is a contradiction. Therefore $(\R, \mathcal{T})$ is not first countable.

    Finally, by Theorem 2.50 $(\R, \mathcal{T})$ can't be second countable
    since it's not first countable.
\end{itemize}
\end{proof}
\cleardoublepage
\begin{proof}{\textbf{2-20}}
    We want to show that second countability, separability, and Lindelöf
    properties are all equivalent for metric spaces.

    Let $(X,d)$ be some metric space which is second countable then by Theorem
    2.50 we know it's Lindelöf and separable.

    Let now $(X,d)$ be a separable metric space then it contains
    a countable dense subset $A$.
    We want to prove the following collection is a basis for $(X,d)$
    \begin{align*}
        \mathcal{B} = \{B_{q}(x): x \in A \text{ and }q \in \Q\}
    \end{align*}
    Then $\mathcal{B}$ is the collection of balls centered at every $x \in A$
    which is countable with radius $q \in \Q$ which is also countable.
    Let $U \subseteq (X,d)$ be an open set then since $A$ is dense there is
    a point $x \in A$ such that $x \in U$ and since the collection of balls
    (of any radius) is a basis for the metric space then there must be some
    $B_{r}(x) \subseteq (X,d)$ such that $B_{r}(x) \subseteq U$.
    So we can take some $q \in \Q$ such that $0 < q < r$ and some
    $y \in B_q(x)$ such that
    $y \in B_{q}(x) \subseteq B_{r}(x) \subseteq U$
    which implies that $\mathcal{B}$ is a countable basis for $(X,d)$
    and thus $(X,d)$ is second countable.

    Let now $(X,d)$ be a Lindelöf metric space then for every open cover of
    $(X,d)$ there is a countable subcover.
    Let us define an open cover as $C_q = \{B_q(x): x \in (X,d)\}$
    where $q \in \Q$ then there is 
    a countable subcover $C_q' \subseteq C_q$. We want to prove the collection
    $\mathcal{B} = \bigcup_{q \in \Q} C_q'$ is a basis for $(X,d)$.
    Let $U \subseteq (X,d)$ be an open set then given some $x \in U$ there
    is $B_{r}(x) \subseteq U$ for some $r \in \R$, let us take some $q \in \Q$
    such that $0 < q < r/2$ then there is some $y \in B_{r/2}(x)$ such that
    $x \in B_q(y) \in \mathcal{B}$. Now let us suppose $z \in B_q(y)$
    we see that $d(y,z) < q < r/2$ but also we know that $d(x,y) < r/2$
    so by the triangle inequality, we have that
    $d(x,z) \leq d(x,y) + d(y,z) < r/2 + r/2 = r$ hence $z \in B_r(x)$ and
    therefore $x \in B_q(y) \subseteq B_r(x) \subseteq U$ which implies that
    $\mathcal{B}$ is a countable basis for $(X,d)$ and thus $(X,d)$
    is second countable. 
\end{proof}
\cleardoublepage
\begin{proof}{\textbf{2-21}}
    We want to show that every locally Euclidean space is first countable.

    We know that $\R^n$ is first countable, and that
    $$\mathcal{B} = \{B_q(x): x \in \R^n \text{ and }q \in \Q\}$$
    is a countable basis for $\R^n$.
    
    Let $M$ be a locally Euclidean space.
    Let us take $p \in M$ with a neighborhood $U \subseteq M$ then since
    $M$ is locally Euclidean there must be some map $\varphi:U \to B_r$
    which is a homeomorphism to an open ball $B_r \subseteq \R^n$.  
    Then $B_r(\varphi(p))$ is a neighborhood of $\varphi(p)$ and since
    $\mathcal{B}$ is a basis for $\R^n$ there is some $q \in \Q$ such that
    $B_q(\varphi(p)) \subseteq B_r(\varphi(p))$ but since $\varphi$ is a
    bijection there is $V_q = \varphi^{-1}(B_q(\varphi(p)))$ such that
    $V_q \subseteq U$ where $V_q$ is a neighborhood of $p$.
    
    Since we can do this for any neighborhood $U$ of $p$ then the collection
    $V_q$ is a countable basis for $p$ and therefore $M$ is first countable.
\end{proof}
\begin{proof}{\textbf{2-23}}
    Let $M$ be a manifold. We know that for every $p \in M$ there is
    $\varphi: U_p \to O_p$ which is a homeomorphism from a neighborhood
    $U_p$ of $p$ to an open set $O_p \in \R^n$ which exists since $M$ is
    locally Euclidean.

    So let us build a collection $\mathcal{B}_p = \{\varphi^{-1}(B_r(\varphi(p))) : r>0\}$
    which is a collection of open sets of $U_p$ which are homeomorphic
    to an open ball in $\R^n$.
    Then we can build a collection $\mathcal{B} = \bigcup_{p \in M} \mathcal{B}_p$.
    We want to prove $\mathcal{B}$ is a basis for $M$.

    So let $V \subseteq M$ be an open set where $p \in V$ and let us consider
    the set $U = V \cap U_p$ then we see that $U$ is open, $p \in U$ and
    $\varphi(U)$ is open in $\R^n$ so there is some ball
    $B_r(\varphi(p)) \subseteq \varphi(U)$ but $\varphi^{-1}(B_r(\varphi(p)))$
    is in $\mathcal{B}$ then we have that
    $$p \in \varphi^{-1}(B_r(\varphi(p))) \subset U \subset V$$
    which implies that $\mathcal{B}$ is a basis for $M$.
    Therefore every manifold $M$ has a basis of coordinate balls. 
\end{proof}
\begin{proof}{\textbf{2-25}}
    Let $M$ be an n-dimensional manifold with boundary and let $p \in \inter M$
    then there is a domain of an interior chart $U \subseteq M$ of $p$ such
    that $\varphi(U)$ is an open set in $\R^n$ where $\varphi$
    is a homeomorphism.
    
    We want to prove first that $U \subseteq \inter M$ by contradiction.
    Let us suppose that there is at least one point $q \in U$ such that
    $q \not\in \inter M$ then $q \in \partial M$ and so $\varphi(q)$ must be
    in $\partial \HH^n$ so $U$ is not the domain of an interior chart which is
    a contradiction and must be that $U \subseteq \inter M$.

    Then $\inter M$ is second countable, a Hausdorff
    space and locally Euclidean therefore $\inter M$ is an n-manifold without
    boundary.

    On the other hand, for each $p \in \inter M$ we have a domain of an interior 
    chart (or neighborhood) $U_p \subseteq \inter M$ then
    $\inter M = \bigcup_{p\in\inter M} U_p$ and since each $U_p$ is open then
    $\inter M$ is open.
\end{proof}
\end{document}
















